%%
%% Beginning of file 'sample61.tex'
%%
%% Modified 2016 September
%%
%% This is a sample manuscript marked up using the
%% AASTeX v6.1 LaTeX 2e macros.
%%
%% AASTeX is now based on Alexey Vikhlinin's emulateapj.cls 
%% (Copyright 2000-2015).  See the classfile for details.

%% AASTeX requires revtex4-1.cls (http://publish.aps.org/revtex4/) and
%% other external packages (latexsym, graphicx, amssymb, longtable, and epsf).
%% All of these external packages should already be present in the modern TeX 
%% distributions.  If not they can also be obtained at www.ctan.org.

%% The first piece of markup in an AASTeX v6.x document is the \documentclass
%% command. LaTeX will ignore any data that comes before this command. The 
%% documentclass can take an optional argument to modify the output style.
%% The command below calls the preprint style  which will produce a tightly 
%% typeset, one-column, single-spaced document.  It is the default and thus
%% does not need to be explicitly stated.
%%


% These are my private definitions

\newcommand\LL{\mbox{$\:\lambda\lambda $ }}
\newcommand\lam{\mbox{$\:\lambda\ $}}
\newcommand\lamlam{\mbox{$\:\lambda\lambda $ }}
\newcommand\ha{{H$\alpha$}}
\newcommand\hb{{H$\beta$}}

%Needs trailing \ except at end of sentence
\newcommand\BT{B_T}
\newcommand\MBT{M_{B_T}}
\newcommand\Mstar{M^*}
\newcommand\Lstar{L^*}
\newcommand\Mpc{\:\rm{Mpc}}
\newcommand\kpc{\:\rm{kpc}}
\newcommand\pc{\:\rm{pc}}
\newcommand\kms{\:\rm{km\,s^{-1}}}
\newcommand\rfid{r_{\rm{fid}}}

\newcommand\LUM{\:{\rm erg\:s^{-1}}}
\newcommand\FLUX{\:{\rm erg\:cm^{-2}\:s^{-1}}}
\newcommand\FLUXARCSEC{\:{\rm erg\:cm^{-2}\:s^{-1}\:arcsec^{-2}}}
\newcommand\COUNTS{\:{\rm cts\:s^{-1}}}
\newcommand\CPM{\:{\rm s^{-1}\:arcmin^{-2}}}
\newcommand\VEL{\:{\rm km\:s^{-1}}}
\newcommand\ETAL{{\it et\:al.}}
\newcommand\etal{{\it et\:al.}}

\newcommand\OIGS{\:{\rm ergs\:cm^{-2}\:s^{-1}\:\AA^{-1}}}
%\newcommand\LA{Lyman\thinspace$\alpha$}
%\newcommand\LB{Lyman\thinspace$\beta$}
%\newcommand\LG{Lyman\thinspace$\gamma$}
%def\LD{Lyman\thinspace$\delta$}

%Use Roman numbers to designate ionization state and first two digits
%of wavelength in Ang. where necessary

\newcommand\OiL{[\ion{O}{1}] $\lambda 6300$}
\newcommand\OiiL{[\ion{O}{2}] $\lambda\lambda 3727, 3729$}
\newcommand\oiiL{[\ion{O}{2}] $\lambda 3727$}
\newcommand\OiiiL{[\ion{O}{3}] $\lambda\lambda 4959,5007$}
\newcommand\SiiL{[\ion{S}{2}] $\lambda\lambda 6717, 6731$}
\newcommand\NiiL{[\ion{N}{2}] $\lambda\lambda 6548, 6583$}
\newcommand\OiLL{[\ion{O}{1}] $\lambda\lambda 6300, 6363$}

\newcommand\sii{[\ion{S}{2}]}
\newcommand\nii{[\ion{N}{2}]}
\newcommand\oi{[\ion{O}{1}]}
\newcommand\oii{[\ion{O}{2}]}
\newcommand\oiii{[\ion{O}{3}]}
\newcommand\feii{[\ion{Fe}{2}]}
\newcommand\hii{\ion{H}{2}}
\newcommand\hi{\ion{H}{1}}
\newcommand\perpix{\: \rm pixel^{-1}}
\newcommand\gal{NGC\,6946}

\newcommand{\EXPN}[2]{\mbox{$#1\times 10^{#2}$}}
\newcommand{\EXPU}[3]{\mbox{\rm $#1 \times 10^{#2} \rm\:#3$}}  % exponent with units
\newcommand{\POW}[2]{\mbox{$\rm10^{#1}\rm\:#2$}}

%private definitions end here

%\begin{document}

\documentclass[modern]{aastex62}

%%
%% using aastex version 6.2

%% The default is a single spaced, 10 point font, single spaced article.
%% There are 5 other style options available via an optional argument. They
%% can be envoked like this:
%%
%% \documentclass[argument]{aastex61}
%% 
%% where the arguement options are:
%%
%%  twocolumn   : two text columns, 10 point font, single spaced article.
%%                This is the most compact and represent the final published
%%                derived PDF copy of the accepted manuscript from the publisher
%%  manuscript  : one text column, 12 point font, double spaced article.
%%  preprint    : one text column, 12 point font, single spaced article.  
%%  preprint2   : two text columns, 12 point font, single spaced article.
%%  modern      : a stylish, single text column, 12 point font, article with
%% 		  wider left and right margins. This uses the Daniel
%% 		  Foreman-Mackey and David Hogg design.
%%
%% Note that you can submit to the AAS Journals in any of these 6 styles.
%%
%% There are other optional arguments one can envoke to allow other stylistic
%% actions. The available options are:
%%
%%  astrosymb    : Loads Astrosymb font and define \astrocommands. 
%%  tighten      : Makes baselineskip slightly smaller, only works with 
%%                 the twocolumn substyle.
%%  times        : uses times font instead of the default
%%  linenumbers  : turn on lineno package.
%%  trackchanges : required to see the revision mark up and print its output
%%  longauthor   : Do not use the more compressed footnote style (default) for 
%%                 the author/collaboration/affiliations. Instead print all
%%                 affiliation information after each name. Creates a much
%%                 long author list but may be desirable for short author papers
%%
%% these can be used in any combination, e.g.
%%
%% \documentclass[twocolumn,linenumbers,trackchanges]{aastex61}

%% AASTeX v6.* now includes \hyperref support. While we have built in specific
%% defaults into the classfile you can manually override them with the
%% \hypersetup command. For example,
%%
%%\hypersetup{linkcolor=red,citecolor=green,filecolor=cyan,urlcolor=magenta}
%%
%% will change the color of the internal links to red, the links to the
%% bibliography to green, the file links to cyan, and the external links to
%% magenta. Additional information on \hyperref options can be found here:
%% https://www.tug.org/applications/hyperref/manual.html#x1-40003

%% If you want to create your own macros, you can do so
%% using \newcommand. Your macros should appear before
%% the \begin{document} command.
%%
\newcommand{\vdag}{(v)^\dagger}
\newcommand\aastex{AAS\TeX}
\newcommand\latex{La\TeX}

%% Reintroduced the \received and \accepted commands from AASTeX v5.2
% \received{July 1, 2016}
%\revised{September 27, 2016}
%\accepted{\today}
%% Command to document which AAS Journal the manuscript was submitted to.
%% Adds "Submitted to " the arguement.
%\submitjournal{ApJ}

%% Mark up commands to limit the number of authors on the front page.
%% Note that in AASTeX v6.1 a \collaboration call (see below) counts as
%% an author in this case.
%
%\AuthorCollaborationLimit=3
%
%% Will only show Schwarz, Muench and "the AAS Journals Data Scientist 
%% collaboration" on the front page of this example manuscript.
%%
%% Note that all of the author will be shown in the published article.
%% This feature is meant to be used prior to acceptance to make the
%% front end of a long author article more manageable. Please do not use
%% this functionality for manuscripts with less than 20 authors. Conversely,
%% please do use this when the number of authors exceeds 40.
%%
%% Use \allauthors at the manuscript end to show the full author list.
%% This command should only be used with \AuthorCollaborationLimit is used.

%% The following command can be used to set the latex table counters.  It
%% is needed in this document because it uses a mix of latex tabular and
%% AASTeX deluxetables.  In general it should not be needed.
%\setcounter{table}{1}

%%%%%%%%%%%%%%%%%%%%%%%%%%%%%%%%%%%%%%%%%%%%%%%%%%%%%%%%%%%%%%%%%%%%%%%%%%%%%%%%
%%
%% The following section outlines numerous optional output that
%% can be displayed in the front matter or as running meta-data.
%%
%% If you wish, you may supply running head information, although
%% this information may be modified by the editorial offices.
\shorttitle{SNRs in NGC~6946}
\shortauthors{Long, Winkler, \& Blair}
%%
%% You can add a light gray and diagonal water-mark to the first page 
%% with this command:
% \watermark{text}
%% where "text", e.g. DRAFT, is the text to appear.  If the text is 
%% long you can control the water-mark size with:
%  \setwatermarkfontsize{dimension}
%% where dimension is any recognized LaTeX dimension, e.g. pt, in, etc.
%%
%%%%%%%%%%%%%%%%%%%%%%%%%%%%%%%%%%%%%%%%%%%%%%%%%%%%%%%%%%%%%%%%%%%%%%%%%%%%%%%%

%% This is the end of the preamble.  Indicate the beginning of the
%% manuscript itself with \begin{document}.
%\let\la=\lesssim            % For Springer A&A compliance...
\let\ga=\gtrsim
%\newcommand\sq{\mbox{\rlap{$\sqcap$}$\sqcup$}}%
%\newcommand\degr{\arcdeg}%
%\newcommand\arcdeg{\mbox{$^\circ$}}%
%\newcommand\arcmin{\mbox{$^\prime$}}%
%\newcommand\arcsec{\mbox{$^{\prime\prime}$}}%
%\newcommand\Sun{\sun}% Sun symbol, "S"
%\newcommand\Sol{\sun}%
%\newcommand\sun{\odot}%


\newcommand{\EXPN}[2]{\mbox{$#1\times 10^{#2}$}}
\newcommand{\EXPU}[3]{\mbox{\rm $#1 \times 10^{#2} \rm\:#3$}}  % exponent with units
\newcommand{\POW}[2]{\mbox{$\rm10^{#1}\rm\:#2$}}
\newcommand{\SING}[2]{#1$\thinspace \lambda $#2}
\newcommand{\MULT}[2]{#1$\thinspace \lambda \lambda $#2}
\newcommand{\CHINU}{\mbox{$\chi_{\nu}^2$}}
\newcommand{\vsini}{\mbox{$v\:\sin{(i)}$}}
\newcommand{\FUSE}{{\it FUSE}}
\newcommand{\HST}{{\it HST}}
\newcommand{\IUE}{{\it IUE}}
\newcommand{\EUVE}{{\it EUVE}}
\newcommand{\einstein}{{\it Einstein}}
\newcommand{\chandra}{{\it Chandra}}

\newcommand\lam{\mbox{$\:\lambda $ }}
\newcommand\lamlam{\mbox{$\:\lambda\lambda $ }}
\newcommand\ha{{H$\alpha$}}
\newcommand\hb{{H$\beta$}}

\newcommand\BT{B_T}
\newcommand\MBT{M_{B_T}}
\newcommand\Mstar{M^*}
\newcommand\Lstar{L^*}
\newcommand\Mpc{\:\rm{Mpc}}
\newcommand\kpc{\:\rm{kpc}}
\newcommand\pc{\:\rm{pc}}
\newcommand\kms{\:\rm{\,km\,s^{-1}}}
\newcommand\asy{\:\rm{\,arcsec\,yr^{-1}}}
\newcommand\masy{\:\rm{\,mas\:yr^{-1}}}
\newcommand\rfid{r_{\rm{fid}}}

\newcommand\arcspix{\:\arcsec\:{\rm pixel}^{-1}}
\newcommand\perpix{\:{\rm pixel}^{-1}}
\newcommand\peras{\:{\rm arcsec}^{-1}}
\newcommand\perassq{\:{\rm arcsec}^{-2}}
\newcommand\LUM{\:{\rm ergs\:s^{-1}}}
\newcommand\FLUX{\:{\rm ergs\:cm^{-2}\:s^{-1}}}
\newcommand\FLUXARCSEC{\:{\rm ergs\:cm^{-2}\:s^{-1}\:arcsec^{-2}}}
\newcommand\FLUXARCMIN{\:{\rm ergs\:cm^{-2}\:s^{-1}\:arcmin^{-2}}}
\newcommand\FLUXSR{\:{\rm ergs\:cm^{-2}\:s^{-1}\:sr^{-1}}}
\newcommand\COUNTS{\:{\rm counts\:s^{-1}}}
\newcommand\CPM{\:{\rm counts\:s^{-1}\:arcmin^{-2}}}
\newcommand\cpksas{\:{\rm cnts\:ks^{-1}\:arcsec^{-2}}}
\newcommand\cpas{\:{\rm cnts\:arcsec^{-2}}}
\newcommand\VEL{\:{\rm km\:s^{-1}}}
%\newcommand\ETAL{{\it et\:al.}}
\newcommand\OIGS{\:{\rm ergs\:cm^{-2}\:s^{-1}\:\AA^{-1}}}
\newcommand\LA{Lyman\thinspace$\alpha$}
\def\LB{Lyman\thinspace$\beta$}
\def\LG{Lyman\thinspace$\gamma$}
\def\LD{Lyman\thinspace$\delta$}
\newcommand\nsnr{NGC\thinspace4449-1}
\newcommand\ngal{NGC\thinspace4449}
\newcommand\sna{SN\thinspace1987A}
\newcommand\peryr{\:{\rm yr}^{-1}}
\newcommand\persec{\:{\rm s}^{-1}}

%A few moreprivate definitions
\newcommand\eg{{\it e.g.}}
\newcommand\ie{{\it i.e.}}
\newcommand\etal{{\it et\thinspace al.}\thinspace}
\newcommand\ergflux{ergs\thinspace cm$^{-2}$\thinspace s$^{-1}$\ }


%These use ion defns are independent of AASTeX macros

\newcommand\sii{[S\,{\footnotesize II]}}
\newcommand\oiii{[O\,{\footnotesize III]}}
\newcommand\hi{H\t,{\footnotesize I}}
\newcommand\hii{H\,{\footnotesize II}}
\newcommand\feii{[Fe\t,{\footnotesize II]}}

\def\BT{B_T}
\def\MBT{M_{B_T}}
\def\Mstar{M^*}
\def\Lstar{L^*}
\def\Mpc{\:\rm{Mpc}}
\def\kpc{\:\rm{kpc}}
\def\pc{\:\rm{pc}}
\def\kms{\:\rm{\,km\,s^{-1}}}
\def\rfid{r_{\rm{fid}}}


\def\LUM{\:{\rm ergs\:s^{-1}}}
\def\FLUX{\:{\rm ergs\:cm^{-2}\:s^{-1}}}
\def\FLUXARCSEC{\:{\rm ergs\:cm^{-2}\:s^{-1}\:arcsec^{-2}}}
\def\FLUXARCMIN{\:{\rm ergs\:cm^{-2}\:s^{-1}\:arcmin^{-2}}}
\def\FLUXSR{\:{\rm ergs\:cm^{-2}\:s^{-1}\:sr^{-1}}}
\def\COUNTS{\:{\rm counts\:s^{-1}}}
\def\CPM{\:{\rm counts\:s^{-1}\:arcmin^{-2}}}
\def\VEL{\:{\rm km\:s^{-1}}}
\def\ETAL{{\it et\:al.}}
\def\OIGS{\:{\rm ergs\:cm^{-2}\:s^{-1}\:\AA^{-1}}}
\def\LA{Lyman\thinspace$\alpha$}

%A few moreprivate definitions
\def\eg{{e.g.\,}}
\def\ie{{\it i.e.}}
\def\etal{et\thinspace al.\ }
%\def\etal{\it{et\thinspace al.}\ }
\def\ergflux{ergs\thinspace cm$^{-2}$\thinspace s$^{-1}$\ }
\def\ha{H$\alpha$}
\def\hb{H$\beta$}
\def\n6946{NGC\,6946}

%\@ifundefined{amsfonts.sty}{\input{amsfonts.sty}}{}%
%\newsymbol\lesssim 132E
%\newsymbol\gtrsim 1326




\usepackage{natbib}
\usepackage{comment}
\usepackage{todonotes}
\usepackage{graphicx}
\begin{document}
%\slugcomment{Draft \today}


\title{A New, Larger Sample of Supernova Remnants in NGC\,6946}

%% LaTeX will automatically break titles if they run longer than
%% one line. However, you may use \\ to force a line break if
%% you desire. In v6.1 you can include a footnote in the title.

%% A significant change from earlier AASTEX versions is in the structure for 
%% calling author and affilations. The change was necessary to implement 
%% autoindexing of affilations which prior was a manual process that could 
%% easily be tedious in large author manuscripts.
%%
%% The \author command is the same as before except it now takes an optional
%% arguement which is the 16 digit ORCID. The syntax is:
%% \author[xxxx-xxxx-xxxx-xxxx]{Author Name}
%%
%% This will hyperlink the author name to the author's ORCID page. Note that
%% during compilation, LaTeX will do some limited checking of the format of
%% the ID to make sure it is valid.
%%
%% Use \affiliation for affiliation information. The old \affil is now aliased
%% to \affiliation. AASTeX v6.1 will automatically index these in the header.
%% When a duplicate is found its index will be the same as its previous entry.
%%
%% Note that \altaffilmark and \altaffiltext have been removed and thus 
%% can not be used to document secondary affiliations. If they are used latex
%% will issue a specific error message and quit. Please use multiple 
%% \affiliation calls for to document more than one affiliation.
%%
%% The new \altaffiliation can be used to indicate some secondary information
%% such as fellowships. This command produces a non-numeric footnote that is
%% set away from the numeric \affiliation footnotes.  NOTE that if an
%% \altaffiliation command is used it must come BEFORE the \affiliation call,
%% right after the \author command, in order to place the footnotes in
%% the proper location.
%%
%% Use \email to set provide email addresses. Each \email will appear on its
%% own line so you can put multiple email address in one \email call. A new
%% \correspondingauthor command is available in V6.1 to identify the
%% corresponding author of the manuscript. It is the author's responsibility
%% to make sure this name is also in the author list.
%%
%% While authors can be grouped inside the same \author and \affiliation
%% commands it is better to have a single author for each. This allows for
%% one to exploit all the new benefits and should make book-keeping easier.
%%
%% If done correctly the peer review system will be able to
%% automatically put the author and affiliation information from the manuscript
%% and save the corresponding author the trouble of entering it by hand.

\correspondingauthor{Knox S. Long}
\email{long@stsci.edu}

\author[0000-0002-4134-864X]{Knox S. Long}
\affil{Space Telescope Science Institute,
3700 San Martin Drive,
Baltimore MD 21218, USA; long@stsci.edu}
\affil{Eureka Scientific, Inc.
2452 Delmer Street, Suite 100,
Oakland, CA 94602-3017}

\author[0000-0001-6311-277X]{P. Frank Winkler}
\affiliation{Department of Physics, Middlebury College, Middlebury, VT, 05753; 
winkler@middlebury.edu}

\author[0000-0003-2379-6518]{William P. Blair}
\affiliation{The Henry A. Rowland Department of Physics and Astronomy, 
Johns Hopkins University, 3400 N. Charles Street, Baltimore, MD, 21218; 
wblair@jhu.edu}



%% Note that the \and command from previous versions of AASTeX is now
%% depreciated in this version as it is no longer necessary. AASTeX 
%% automatically takes care of all commas and "and"s between authors names.

%% AASTeX 6.1 has the new \collaboration and \nocollaboration commands to
%% provide the collaboration status of a group of authors. These commands 
%% can be used either before or after the list of corresponding authors. The
%% argument for \collaboration is the collaboration identifier. Authors are
%% encouraged to surround collaboration identifiers with ()s. The 
%% \nocollaboration command takes no argument and exists to indicate that
%% the nearby authors are not part of surrounding collaborations.

%% Mark off the abstract in the ``abstract'' environment. 
\begin{abstract}

The relatively nearby spiral galaxy NGC~6946 is one of the most actively star forming galaxies in the local Universe.  Ten supernovae (SNe) have been observed since 1917, and hence NGC~6946 surely contains a large number of supernova remnants (SNRs).  
Here we report a new optical search for these SNRs in NGC~6946 using narrow-band images obtained with the WIYN telescope.  We
identify 147 emission nebulae 
as likely SNRs,  based on elevated \sii:\ha\ ratios compared to \hii\ regions.  We have obtained spectra of 102 of these emission nebulae with Gemini North--GMOS; of these, 89 have \sii:\ha\ ratios greater than 0.4.  There is very little overlap in our sample between the SNR candidates identified  by Lacey et al. (2001) from radio data.  Very few of the SNR candidates are known X-ray sources, unlike the situation in some other galaxies such as M33 and M83.  The emission line ratios, e.g., \nii:\ha, of the candidates in NGC~6946 are typical of those observed in SNR samples from other galaxies with comparable metallicity.  
None of the  candidates observed in our low resolution spectra show evidence of anomalous abundances or  significant velocity broadening.   We also obtained new spectra of two of the historical SNe in NGC~6946, SN1980K and SN2004ET, both of which show high velocity ejecta. We did not detect emission from any of the other historical SNe.

\end{abstract}

%% Keywords should appear after the \end{abstract} command. 
%% See the online documentation for the full list of available subject
%% keywords and the rules for their use.
\keywords{galaxies: individual (NGC~6946) -- galaxies: ISM  -- supernova remnants}

% \todo[inline]{WPB: This will likely have to get shortened further before submission.}

%% From the front matter, we move on to the body of the paper.
%% Sections are demarcated by \section and \subsection, respectively.
%% Observe the use of the LaTeX \label
%% command after the \subsection to give a symbolic KEY to the
%% subsection for cross-referencing in a \ref command.
%% You can use LaTeX's \ref and \label commands to keep track of
%% cross-references to sections, equations, tables, and figures.
%% That way, if you change the order of any elements, LaTeX will
%% automatically renumber them.

%% We recommend that authors also use the natbib \citep
%% and \citet commands to identify citations.  The citations are
%% tied to the reference list via symbolic KEYs. The KEY corresponds
%% to the KEY in the \bibitem in the reference list below. 

\section{Introduction} \label{sec:intro}

\gal\ is a nearby \cite[$6.72\pm 0.15$ Mpc,][]{tikhonov14}, nearly face-on \cite[$i = 32.6\degr$,][]{deblok08} galaxy with four flocculent spiral arms.  The galaxy is currently undergoing a major starburst, and as a result it has been the site of ten historical supernovae (SNe) since 1917, the most of any known galaxy. According to \citet{jarrett13} the total star formation rate is $3.2\, M_\sun\,{\rm yr}^{-1}$, a high rate that is thought to be bar-driven.  A total of 121 bubbles, probably created by stellar winds and multiple SNe at the star-formation sites, have been identified in \hi\ gas that extends well outside the bright portions of the optical galaxy \citep{boomsma08}.  %The high SFR in \gal\ is thought to be bar-driven.   
Given these properties, one expects that a large number of supernova remnants (SNRs) should be present in \gal, since SNRs tend to remain visible for thousands of years.   

Optically, SNRs are usually identified on the basis of high \sii:\ha\ ratios compared to \hii\ regions.  In bright \hii\ regions, most S is found in the form of S$^{++}$ or S$^{+++}$, and as a result the \sii:\ha\ ratios are typically 0.1 or smaller.  In SNRs, where emission is driven by impulsive heating from shock waves, S$^+$ is found in an extended recombination and cooling zone behind the shock, and the \sii:\ha\ ratios are typically $\gtrsim0.4$.  The diagnostic can become less deterministic as one pushes to lower surface brightnesses, as recently discussed by \cite{long18} for the case of M33.

The first optical search for SNRs in \gal\ was made by \cite{matonick97}, hereafter MF97, who used interference filter imagery to identify 27 emission nebulae with \sii:\ha\ ratios $\geq$ 0.45 as SNRs.  One of these sources, MF97-16, was later associated with the ultraluminous black hole X-ray binary \gal\ X-1 \citep{roberts03}. Though very rare, such ultra-luminous X-ray sources (ULXs) have  the hard X-ray spectra that produce line ratios in the surrounding circumstellar ISM that resemble those expected from SNRs. To our knowledge, no other optical searches for SNRs in NGC~6946 exist, nor have spectra of the remaining MF97 objects ever been reported.  

% \hii\ regions and SNRs also radiate at radio wavelengths.  SNRs typically have steep non-thermal radio spectral indices arising from synchrotron radiation, whereas \hii\ regions have relatively flat thermal spectra arising from free-free radiation.  At radio wavelengths, SNRs in nearby galaxies are generally identified as non-thermal radio sources associated with optical \ha\ emission.  (This association criterion arises because many background radio sources---typically active galaxy nuclei (AGNs)---also have non-thermal spectra, and like AGNs, the SNRs are generally unresolved in radio images of galaxies beyond the Local Group.)  In the case of \gal,  \cite{lacey01} identified 35 radio-selected SNR candidates on the basis of these criteria, none of which overlap with the MF97 list of optical candidates.  Their radio candidates were located in the brightest parts of the spiral arms of \gal; Lacey \& Duric suggested that the reason the two lists do not overlap is primarily due to selection effects.

Here we discuss a new, more sensitive optical search for SNRs in \gal\ in which we identify a total of 147 SNR candidates using interference filter imagery.  We also discuss spectroscopic observations of 102 of these candidates, which we use to verify the ratios obtained from the imaging and to characterize other characteristics of our new optical sample.  The paper is organized as follows:  Section 2 describes both our imaging and spectroscopy observations,  presents our significantly expanded catalog of optical SNR candidates, and the results of our spectroscopy.  Section 3 discusses these results in the context of both \gal\ and other spiral galaxies, and Section 4 provides a brief summary of our results. 


\section{Observations and Data Reduction \label{sec:observations}}
\subsection{Imaging and Catalog of SNR Candidates}

We carried out narrow-band imaging observations of \gal\ from the 3.5m WIYN telescope and MiniMosaic imager on Kitt Peak on the nights of 2011 June 26-28 (UT).\footnote{The WIYN Observatory is a joint facility of the University of Wisconsin-Madison, Indiana University, the National Optical Astronomy Observatory and the University of Missouri.}  The so-called ``Minimo'' was mounted at the f/6.3 Nasmyth port and consisted of a pair of $2048\times4096$ SITe chips, with a field  9\farcm 6 square at a scale of 0\farcs 14 pixel$^{-1}$.  We used interference filters that pass lines of \ha, \sii \lamlam6716,6731, and \oiii \lam 5007, plus red and green narrow-band continuum filters so we could subtract the stars and produce pure emission-line images. Frames in each filter were dithered to enable automatic removal of cosmic rays and bad pixels.  Further observational details are given in Table 1.  It is noteworthy that the \ha\ filter was quite narrow in bandwidth, 27\ \AA\ FWHM. Its transmission is 69\% at the rest wavelength of \ha, but only 11\% at 6548\ \AA\ and 16\% at 6583\ \AA; hence, the \nii\ lines are greatly attenuated relative to \ha.\footnote{The recessional velocity of \gal\ is only $40 \kms$, so the lines are redshifted by $\lesssim 1$ \AA.}  This facilitates identification of SNRs based on their image-derived \sii:\ha\ ratios.  Seeing throughout this run was about $1\arcsec \approx 32$ pc at the distance of \gal.  While sufficient for identifying SNRs in all but the most crowded regions, this resolution was insufficient for obtaining properties such as diameter or morphology.  


We used standard IRAF\footnote{IRAF is distributed by the National Optical Astronomy Observatory, which is operated by the Association of Universities for Research in Astronomy, Inc., under cooperative agreement with the National Science Foundation.} techniques for processing the images, including overscan correction, bias subtraction, and flat-fielding using dome flats.  Procedures in the IRAF {\texttt mscred} package were then used to combine the data from the individual chips into a mosaic image for each frame, assigned a WCS for each using stars from the USNOB1 catalog \citep{monet03}.  We then stacked  all the images by filter onto an arbitrary standard coordinate system with a scale of 0\farcs 20 pixel$^{-1}$ and scaled and subtracted the continuum images from the emission-line ones (red from \ha\ and \sii; green from \oiii) to give pure emission-line images with most of the stars and galaxy background light removed.  Finally, we used observations of several spectrophotometric standard stars from the catalog of \citet{massey88} to flux calibrate the emission-line images.  Figure \ref{fig_rgb_overview} shows a color version of the final images (R = \ha, G = \sii, B = \oiii), with the positions of historical SNe in \gal\ also indicated. 


%\todo[inline]{Bill is responsible for one or more enlargements of a smaller region in a multi-panel format, showing the subtracted data.  He is also responsible for the associated text.  WPB: Both of these (figure and text) have been done in the section below.} 

To select SNR candidates, we used the DS9 display program to show the continuum-subtracted WIYN images in all three emission lines as well as a \sii:\ha\ ratio image and a continuum image (to discriminate stars or stellar subtraction residuals from point-like nebulae).   We then visually inspected these to select SNR candidates based on a high \sii:\ha\ ratio.  The initial inspection was carried out by Middlebury undergraduate Marc DeLaney; subsequently two of us (WPB and PFW) compiled independent candidate lists; we then conferred to agree on a consensus list of 147 candidates, including the 27 from MF97. The positions of all 147 candidates are shown in Figure ~\ref{fig_index_overview}. The vast majority of the SNR candidates appear within the spiral arms or clustered on the outskirts of large complexes of \hii\ emission and star formation.

Figure \ref{fig_diagnostic} shows an example for a small region $\sim$1.5\arcmin\ south of the nucleus (see Fig. ~\ref{fig_index_overview}) containing two of the MF97 objects and several new SNR candidates. In this example we have combined the three continuum-subtracted emission line images into a single color panel that shows how the stronger \sii\ and/or \oiii\ emission from the SNR candidates makes them stand out.  The \sii:\ha\ ratio image was a key diagnostic for drawing our eyes to potential objects of interest. Then an assessment of the integrity of the candidate as an emission object was made by inspection of the actual images.  This was done to separate candidates from stellar residuals or false regions of higher ratio in the ratio map that were due to noise.

Having performed similar SNR searches in other galaxies such as M33 and M83, it is worth noting some differences for \gal.  M33 is of course much closer, and the nebulae of interest are almost always resolved.  M83 is much more distant (4.6 Mpc) than M33, but our search there was aided by the exceptional seeing conditions we obtained at the Magellan telescope (0.5\arcsec) \citep{blair12}.  Many objects were resolved, but others extended down to the limits of what even HST could resolve ($\sim$1-2 pc) \citep{blair14}.  Our WIYN survey of \gal\ did not have exceptional seeing, and the distance is some 50\% larger than for M83.  Hence, relatively few of the nebulae of interest are resolved, and with variations in the complex galaxy background, it is much more difficult to perform a systematic search.  The use of the \sii:\ha\ ratio image was particularly helpful for \gal, which is the most distant galaxy for which we have performed this kind of ground-based SNR survey. Nonetheless, while we have expanded the SNR candidate list very substantially, clearly completeness has not been achieved.  Our list of 147 SNR candidates falls far short of M83 for example (with 300+), even though the SN rate is larger in \gal.  Higher resolution data (better seeing and/or HST imaging) would no doubt help substantially, but the greater distance for \gal\ is still a limiting factor.

In addition, we initially selected 51 emission nebulae with relatively high \oiii:\ha\ ratios (using an \oiii:\ha\ ratio image in the display).  Most, if not all, of these nebulae were expected to be \hii\ regions but we hoped that one or more of these regions might be a young, ejecta-dominated SNR, simlar to Cas A in the Galaxy. With ten SNe in the last century and a high incidence of massive stars, one might expect a number of young, ejecta-dominated SNRs to be present.  Though none of the O-selected candidates had ratios as extreme as seen for Cas A or 1E0102-72.3 in the Small Magellanic Cloud, we nevertheless selected some of these for follow-up spectroscopy.  None of these nebulae for which we obtained spectra have any indication of ejecta in their spectra or the broad lines would expect from a very young SNR.  All are \hii\ regions with somewhat enhanced ionization state, and so these form the bulk of the \hii\ sample we use for comparison with the SNR candidates sample below. Finally, we inspected the positions of the nine\footnote{As of the 2011 observations there were nine SNe; a tenth SN was recorded in 2017.} historical SNe in \gal\ for evidence of nebular emission; we only detected line emission at the positions of SN1980K and SN2004et, both of which we targeted for follow-up spectroscopy. (See section 3.4 below.)

%\todo[inline]{ksl - It might be worthwhile to have a few sentences of comparison here to our work on M83 where the Magellan seeing made if far easier to see extended emission, and in M33 of course where all of the SNRs are resolved.  This is about the limit of what we can do from the ground, particularly with the seeing we had with WYIN.  To do better, one needs better seeing.  WPB: I tried to incorporate these points into the text added above with the example diagnostic figure.}

The SNR candidates are listed in Right Ascension order in Table \ref{table_candidates}.  For each candidate, we provide (1) a source name, (2,3) the position  (J2000), (4) the deprojected galactocentric distance (GCD), (5) the \ha\ flux as derived from the emission line images, (6) the \sii:\ha\ ratio measured from the images, (7) the spectrum we used to confirm the imaging ratios (see below), (8) whether or not the object has a spectroscopic confirmation that the  \sii:\ha\ ratio is $\ge$ 0.4 (see below) and (9) other names for the source.

% \todo[inline]{We need a consistent story about the HII regions.  We mention that we selected some O3 objects.  We got spectra of 24 of these.   We have another 41 spectra of HII regions along the slit.  We got very few HII regions just by chance, though these are included in the 41.  It's hard to actually parse out the history of which are which.  WPB: Note that I added some words above in the ``[O III] object" discussion to at least partially address this concern.}

\subsection{Spectroscopy; Emission-Line Fluxes}

%We used a combination of procedures in the GMOS and apextract packages of IRAF to reduce the spectra.  

We used the Gemini Multi-Object Spectrograph (GMOS) on the 8.2m Gemini-North telescope to obtain all the spectra reported here, during queue-scheduled programs in semesters 2014B (program GN-2014B-Q-83) and 2015B (program GN-2015B-Q-91).  
For the 2014B program, we designed six custom masks, each with 20-30 slitlets targeting SNR candidates whose positions we determined from our 2011 WIYN images, together with  short $R$-band pre-images of several \gal\ fields taken with GMOS earlier in 2014 as part of the spectroscopy preparation program.    
We used two additional masks (which we refer to as masks 7 and 8 for simplicity)  for the 2015B program.  Slitlets in one or more of our eight masks were placed on 102 distinct SNR candidates, including ones with a range of sizes, GCDs, and ISM environments (locations in arms and in 
inter-arm regions).  In addition to the SNR candidates, we also placed a number of slitlets on \hii\ regions for comparison purposes in both 2014 and 2015.
% Next sentence redundant with below, so commented out here.  WPB
%Figure \ref{fig_index_overview} shows all the SNR candidates from the sample for which we obtained spectra (red boxes). 

For all the spectra we used the 600 line\,mm$^{-1}$ grating designated G5307 and a GG455 cut-off filter to block second-order spectra.   The detector in both years was a mosaic of three e2v deep-depletion CCD chips, binned by 2 in the spatial direction (for a scale of 0\farcs 146 pixel$^{-1}$) and by 4 in the dispersion direction.  
The dispersion was 1.84 \AA\,  pixel$^{-1}$ (binned), resulting in coverage of the spectral range from at least \hb\ through \sii\lamlam 6716,\,6731 for virtually all the objects.\footnote{The detailed wavelength coverage for individual objects naturally varied with slitlet position in the dispersion direction.}  Our  masks had  slitlet  widths from 1\farcs 25 to 1\farcs 75, with wider slits used for the larger objects, and lengths of 6\arcsec\ or longer to permit local background and sky subtraction.   

With each mask, we took spectra at three or four slightly different grating tilts, to cover wavelength gaps between chips and to gain somewhat more total spectral range.\footnote{An exception was mask 6, done late in the 2014B semester, for which our full set of planned observations were never completed.}  At each wavelength setting, we obtained two or more identical exposures to minimize the effects of cosmic rays.
For calibration, we programmed quartz flats and CuAr arc frames immediately before or after the science exposures with each mask and grating setting.  A journal of all the science observations from both 2014 and 2015 appears in Table \ref{obs_log}.  The SNR candidates for which we obtained spectra are indicated by the small red boxes in Fig.~\ref{fig_index_overview}.

%\todo[inline]{WPB: Next paragraph: Frank, do we need to say something about how the separate grating offset exposures were shifted and co-added prior to obtaining the final 2-D spectra for each slitlet?}

The data were processed using standard procedures  from the {\tt gemini} package in IRAF  for bias subtraction, flat-fielding, wavelength calibration, and combination of spectra with different grating tilts to provide the final results.  
Flux calibration was based on baseline GMOS observations of a few spectrophotometric standard stars, carried out in the same semester as part of standard GMOS operations.  
%  IRAF footnote already given above, so I commented this out.  WPB
%\footnote{IRAF is distributed by the National Optical Astronomy Observatory, which is operated by the Association of Universities for Research in Astronomy, Inc., under cooperative agreement with the National Science Foundation.}  
During the processing, the 2-D spectra from different slitlets were separated to give individual 2-D spectra from each slitlet.   We examined each of these individually and selected the object region, as well as one or more sky background regions,  stripped out 1-D spectra of each, and then subtracted the sky spectrum from the corresponding object to obtain the final background-subtracted object spectra.  
Many of the objects are located in regions with bright surrounding galactic background (both continuum and emission lines) from NGC\,6946, so the selection of a representative local background was done on a best-effort basis.    In addition to the targeted SNR candidates and \hii\ regions, we extracted spectra from other \hii\ regions which appeared by chance in the slits when this was possible.   This allowed us to increase our sample of \hii\ regions from 24 to 65.
We then performed  fits to obtain emission line fluxes  from the spectra, assuming Gaussian profiles, for the following lines and line complexes:  \hb\ alone,  the \oiii\ doublet,  the \oi\ doublet,  the \ha-\nii\ region, and   the \sii\ doublet. For the fits, we assumed that the background varied linearly with wavelength around each line, and that the FWHM of all lines in each complex was the same.

Representative examples of the spectra that were obtained are shown in Fig.\ \ref{fig_example_spectra}.   The three SNRs were selected primarily to show how the quality of the spectra changes as a function of brightness.  The spectroscopically obtained \sii:\ha\ ratio for L19-048 was 0.45, just above the value for spectroscopic confirmation, while those for the previously known bright object L19-097=MF-015 and the faint candidate L19-096 are higher.  All three SNR candidates show evidence of emission from [O I], which is another indicator that the emission we see arises from shock-heated gas. 

Table \ref{table_snr_spec}  lists the information we obtained for the SNR candidates for which we obtained spectra.  Specifically we list (1) the source name, (2) the extracted  \ha\ flux, (3-9) ratios of various emission lines to \ha\ [taken to be 300], (10) the total \sii:\ha\ ratio and (11) the measured FWHM of the lines in the \ha-\nii\ complex.  For doublets, where the line ratio is constrained by atomic physics, that is \oiii, \oi, and \nii, we have listed only the stronger line.  We visually inspected all of the spectra and the fits to them; values which we judge to be more uncertain are indicated with tildes in the Table.  No allowance has been made for additional errors associated with difficulties in background subtraction.  A number of the objects were observed with more than one mask, sometimes with different slit orientations.  In these cases, we used the spectrum which we judged to be the most accurate and report it for reference in the `Spectrum' column in Table \ref{table_candidates}, where (for example) 05.18 should be interpreted as `mask 5 slitlet number 18.'


\section{Analysis and Discussion}

Of the 147 candidate SNRs from the WIYN interference images, we obtained spectra of 102. The spectra were needed to improve our confidence that the \sii:\ha\ ratios of the emission nebulae were indeed high and to begin to characterize the SNRs using accurate line ratios.  The spectroscopic ratios are shown as a function of \ha\ flux in the left panel of  Fig.\ \ref{fig_s2_ha}. \hii\ regions, including both those selected for their \oiii\ emission and those that appeared serendipitously along spectral slits, are shown in blue, while nominal \sii:\ha\ candidates from imaging are shown in red.   Conventionally, emission nebulae are identified as SNRs optically if the \sii:\ha\ ratio exceeds 0.4; 89 of the 102 SNRs with spectra satisfy this criterion, and so we regard these as confirmed SNRs (and they have been listed as such in Table \ref{table_candidates}). 

Clearly, given uncertainties in derived line ratios for faint emission line objects, a dividing line of 0.4 is somewhat arbitrary, and objects just above (or just below) this ratio should be judged with more context.  For example, slightly under-subtracting contaminating \ha\ emission in the spectra could readily explain why some candidates ended up below the threshold in the spectroscopic analysis. We have inspected the objects listed with imaging ratios above the threshold and spectral ratios below, and indeed, many of them are located in regions of \hii\ contamination. Likewise, a slight over-subtraction of \ha\ could enhance the \sii:\ha\ ratio derived, potentially pushing some objects above the threshold.  This is likely the reason why some of the faint \hii\ regions observed spectroscopically (blue symbols in Fig.\ \ref{fig_s2_ha} left) actually lie above the 0.4 threshold.  The observed tendency to see higher \sii:\ha\ ratios closer to the nucleus is likely to be at least partly an abundance effect; similar trends are seen in M33 \citep{long18} and in M83 \citep{winkler17}. 

There are 45 objects without spectra, so which of these are actually SNRs is uncertain. Given the generally good agreement between imaging and spectral ratios, those objects with imaging ratios well in excess of 0.4 are likely to be good candidates. As shown in the right panel of Fig.\ \ref{fig_s2_ha}, if anything the spectroscopically-determined ratios tend to be higher than the ratios determined from narrow band imaging, and so most of the objects without spectra are also likely to be SNRs.  The somewhat higher spectroscopic ratios are to be expected, since even with our relatively narrow \ha\ filter, some emission from \nii\ was  also passed.   

\subsection{Comparison to MF97}

Of the 27 SNR candidates identified by MF97, our WIYN images also show \sii:\ha\ ratios that exceed 0.4 for all 26 that lie within our field coverage.  MF97 obtained spectra of six of their candidates.  We have obtained spectra of 23 of the MF97 objects, including new spectra of four objects for which MF97 had spectra -- MF003, MF021, MF022, and MF026.  All of these indeed have measured spectroscopic ratios that exceed 0.4.  We note that MF97 used a relatively conservative value of \sii:\ha\ $\ge$ 0.45 to establish their catalog (to avoid the issue of errors in the determined ratio affecting objects just above or below the normal 0.4 criterion).  Hence, it is perhaps not surprising that the previous objects are strongly confirmed here.  MF97 report ``typical 1.5\arcsec\ seeing'' for their work. (MF97 looked at a number of galaxies and they do not give a specific value of seeing for \gal.) Hence, it is also clear that MF97 were only able to find a combination of objects that were relatively bright(so not smeared out by seeing) and/or were relatively isolated from contaminating emission.  They estimated that at least four times as many SNRs were likely present in \gal, and our current survey has far surpassed that estimate.

%\todo[inline]{We need a more complete discussion of MF97 and our results. (WPB-DONE)}
Despite the fact that we have identified many more candidates than , it is clear that even our survey with better seeing and better sensitivity is limited and likely significantly incomplete. Fig. \ref{fig_diagnostic} shows two objects, L19-067 and L19-075 that are in close proximity to \hii\ contamination but we were able to identify as candidates while MF97 could not.  It is not hard to imagine any number of additional objects that have been missed in even more complicated regions of emission, of which there are many.



%\todo[inline]{MF-016 probably needs several sentences as it is a unique object.  WPB: Done (see next).}

The object MF-016, listed here as L19-098, deserves separate mention as it is far and away the brightest object in our catalog.  Originally thought to be a possible example of an exceedingly bright (and possibly multiple) SNR \citep{{blair94},{blair01}}, similar to the extraordinary SNR in NGC~4449 \citep{{blair83},{dannym08}}, X-ray variability was subsequently established that clearly indicates the presence of an accreting black hole binary within the nebular complex \citep{{roberts03}, {fridriksson08}, {rao10}}.  
Most recent analyses \citep{{kaaret10},{ berghea12}}  model the system based on the ULX binary only, but clearly the system involves some combination of shock-heated and X-ray photoionized emission.  \citet{dunne00} show resolved line profiles on the bright emission lines indicating kinematic motions of order 250 -- 400 $\rm km ~ s^{-1}$, and while \citet{roberts03} show the bulk of the X-ray emission is likely due to the ULX, they estimate the SNR component could be as bright as $\rm \sim ~ 2.5 ~ \times ~ 10^{38} ~ ergs ~ s^{-1}$ in X-ray, which is quite substantial for a SNR.  
Although jets are often invoked for accreting ULX binaries, the HST images of the nebula \citep{blair01} are not obviously consistent with this idea; the morphology shows a multiple loop structure, and the likely optical counterpart of the ULX is not centered in the smallest, brightest loop.  Hence, the idea that the complex involves something more complicated than a single SN that created the ULX binary may still be relevant to consider in understanding the overall characteristics of this intriguing object.

\subsection{Spectroscopic properties of the SNR candidates}

As shown in Fig.\ \ref{fig_s2_ratio}, the density-sensitive \sii\ ratio  $\lambda$6717:$\lambda$6731 clusters around the low-density limit of 1.4 for the SNR candidates, and  the fact that  about as many objects have non-physical ratios above 1.4 as below suggests that a) the ratios for some faint objects have significant errors (not unexpected), and b) likely  almost all the objects are close to this limit. This is in contrast to the situation in M83 \citep{winkler17} and to a lesser extent in M33 (L18) where a significant number of the SNRs show evidence of high densities, especially for smaller diameter objects. In the absence of good SNR diameters here, we cannot search for trends with diameter, but upcoming {\em Hubble Space Telescope} observations should provide accurate diameter information for many of these objects.  These {\em HST} images could also reveal the presence of very small SNR candidates, perhaps with high densities, which eluded detection in our ground-based images.

Fig.\ \ref{fig_reddening} shows the observed \hb:\ha\ ratios of the SNR candidates with spectra as a function of GCD. Nearly all of the SNRs show significant reddening, as one would expect since, at 12\degr\ from the Galactic plane, foreground reddening from within our Galaxy along the line of sight is expected to be E(B-V) = 0.29 \citep{schlafly11}.  There is clearly substantial internal and differential reddening  within \gal\ as well, as there is a very evident trend for objects near the center of \gal\ to be more reddened than those more distant from the nucleus.


%\todo[inline]{Frank and Bill have suggested that we do the comparison to models figures in the same way we did for M33 in L18, which includes plots not only of the models but also of where these ratios lie in other galaxies. See Figures 10 and 11 of L18. Bill also thinks the ratios are less than in M83 and wonders why that might be.  WPB: I have added some additional comments below, and am retiring this comment. }




% \todo[inline]{Bill and Frank point out that various other diagrams with GC radius are missing, and in particular \sii:\ha\ ratio vs. \ha\ flux.  If included, Bill or Frank should provide words, and comparison to other galaxies.}

Our SNR line ratios show a  trend of decrease with increasing GCD as seen in Fig. \ref{fig_metal_ratios}, although the dispersion at any particular GCD distance is large.  The trend could well be indicative of abundances gradients in N and S, with the dispersion being due to varying shock conditions and/or or local abundance variations.  However, the trend stands in contrast to SNRs in M33, where both the \nii:\ha\ and \sii:\ha\ ratios have a large range and do not decrease systematically with increasing GCD (although the line ratios in the \hii\ region sample appear well behaved--see L18 Fig. 8).   There is a good correlation between the \nii:\ha\ and \sii:\ha\ ratios of the various objects, as shown in  Fig. \ref{fig_s2_n2}, as also seen in other galaxy samples (cf. Winkler et al. 2017, Fig. 9).

\todo[inline]{WPB: Fig.\ref{fig_s2_n2} is redundant with Fig. \ref{fig_model} -- right panel, where the same ratio is shown on the model grid, so I have removed it from text.  When you guys have seen this and agree, just comment out this ``todo". ksl - I agree they plot the same thing but on vastly different scales.  I don't think it is obvious from Figure 10 that two have a good correlation, so we should discuss what we want to do here.  If we do keep Fig 9, we need to move the paragraph above before the theory discussion. ksl - I have rearranged all of this so its in the propoer order.  I have left Fig 9.  If the consensus is to remove this, then we just need to remove the figure and the sentence above.}

Line ratios in spectra of SNRs are expected to vary both as a function of shock conditions and metallicity.  To see where the SNRs in \gal\ lie, we have compared the line ratios calculated from models by \cite{allen08} using the MAPPINGS III code for a range of shock velocities (100 - 1000 $\VEL$) and pre-shock magnetic fields (10$^{-4}$ - 10 $\mu$G).  The results are shown in Fig. \ref{fig_model}.  The results both for the ratio of [O~III]5007:\hb\ and for \sii:\ha\ fall squarely into the region of the solar metallicity models. This is consistent with expectations for previous abundance studies of \gal\ such as \citet{cedres12}, \citet{gusev13}, and references therein, depending of course on adopted method of determining abundances from strong-line data only. Both of these papers also show a very modest abundance gradient in the \hii\ regions of \gal, with considerable scatter about the mean, although very few \hii\ regions are sampled in the inner portion of the galaxy (cf. {Cedr{\'e}s} et al. Fig. 18).

\subsection{SNRs in Other Wavelength Bands}

Searches for SNRs in \gal\ have been carried out in several other wavelength ranges.  In particular,  \cite{lacey01} identified 35 radio sources as SNR candidates on the criteria that these sources had non-thermal spectral indices and were positionally coincident with \ha\ emission.  As one might expect, these objects are located mostly in the spiral arms of \gal\ where there is active star formation and where one might expect SNe to explode.  There are seven emission nebulae in our list of SNR candidates that lie within 2\arcsec\ of radio SNR candidates; this is five more than had been identified previously, but a small fraction of the total number of radio objects.  \cite{lacey01} argued that the reason that few radio SNRs are detected optically is that the bright \ha\ emission from H II regions makes optical searches for SNRs less sensitive in the spiral arms than in the rest of the galaxy.  The fact that more of the optical SNRs are not detected at radio wavelengths is most likely a question of sensitivity.  All of the SNR candidates identified by \cite{lacey01} have radio luminosities of at least one-tenth of Cas A, much greater than the bulk of SNRs known in the Galaxy.  By contrast, in M33, where \cite{white19} have recently conducted very deep radio map with the Jansky Very Large Array, more than three-quarters of the optically identified SNRs have been detected at radio wavelengths.

As an alternative to using the \sii:\ha\ ratio to identify SNRs in an external galaxy, \hii\ regions are expected to have little if any \feii\ 1.64$\mu$m emission, while \feii\ should be elevated in shocked gas for the same reason that \sii\ is.  Hence, detection of an emission nebula with \feii\ is a strong indication of shock heating. In M83, where HST WFC3 IR imaging in \feii\ is available \citep{blair14}, about 40\% of the optical SNRs in the observed region were detected in \feii, and a handful of compact \feii\ nebulae in particularly dusty regions were declared to be SNR candidates whose optical emission was extincted. This raises the possibility that \feii\ might be able to not only help confirm optical SNR candidates, but actually help to obtain an more complete sample.

In \gal, \citet{bruursema14}  carried out ground-based interference filter imagery in the light of [Fe~II]1.64 $\mu$m. Ground-based \feii\ imaging is difficult due to sky contamination, as noted by  \citet{bruursema14}, but they were able to identify 48 candidate objects that they felt were above the noise.  Interestingly, only three of these objects align with SNR candidates in our sample, L19-076, L19-095=MF-014, and the exceedingly bright ULX MF-016 = L19-098 (discussed above). We are thus left wondering whether the other \feii\ objects are possible SNR candidates or whether the data quality issues are responsible for the large difference between \gal\ and M83. We can say, however, that the bulk of the \citet{bruursema14} candidates are not seen in projection onto the dustiest regions. Ground-based \feii\ observations on M33 appear to be intermediate between these two extremes, where a solid handful of optical SNRs have been clearly detected \citep{morel02}.  Clearly high spatial resolution HST WFC3 IR observations would make a large impact in clarifying the situation for a galaxy as distant as \gal.

%\todo[inline]{Bill, could you add something here, and in particular could you add something indicating what fraction of the SNRs in M83 were detected in [Fe II].  We might want to add something about M33 as well.  WPB: DONE}

SNRs are also X-ray sources and therefore we have also looked to see how many X-ray sources in \gal\ could be found in our candidate lists.  The most detailed X-ray study of \gal\ to date was carried out using {\em Chandra} by \cite{fridriksson08}, who constructed a catalog of 90 point sources, of which 25 appeared to be time variable (and hence likely X-ray binaries).  Of the 90 point sources, there are eight which are positionally coincident with objects in our sample.  Most of these have relatively soft X-ray spectra based on the X-ray hardness ratios reported by \cite{fridriksson08}, consistent with that expected for thermal emission from a SNR.  The main exception is F08-08, which is identified with L19-029, which has hardness ratios that are in the region where X-ray binaries and background galaxies are found.  Additionally, one of the coincident X-ray sources is the ULX L19-098=MF-016.  These are the only two X-ray sources coincident with SNR candidates that also show, according to \cite{fridriksson08} evidence of (long-term) variability. 

Higher percentages of the optically identified sources have been detected in M33 \citep[112/200,][]{long18} and M83 \citep[67/225,][]{long14}.  Of these, M83 is the more relevant for comparison.  M83, a relatively face-on grand design spiral, has a star formation rate of 3-4 $M_\sun\,{\rm yr}^{-1}$ \citep{boisser05}, similar to \gal, but lies a distance of 4.61 Mpc \citep{sahu06} compared to 6.7 Mpc for NGC6946.  M83 was observed for 725 ks with {\em Chandra}, compared to a total of about 175 ks for \gal.    In addition, \gal\ is relatively close to the galactic plane and as a result Galactic absorption reduces the X-ray sensitivity particular below 1 keV.  Indeed, the hydrogen column density along the line of sight to M83 is  \EXPU{4}{20}{cm^{-2}}, whereas as for \gal\ it is \EXPU{1.8}{21}{cm^{-2}} \citep{kaberla05}. For a thermal plasma with an effective temperature of 0.6 keV, the combination of extra distance and more absorption implies that a SNR in \gal\ would have about 1/3 the count rate of one in M83.  Consequently, it is not surprising that we have detected fewer SNRs in X-rays in \gal\ than in M83.

%\todo[inline]{One should probably look further into the X-ray coincidences.  How many of the identified sources were time variable.  Also, MF-016 which is one of the sources is a special case. It would also be sensible to see whether the radio coincidences were particularly bright in the radio or the optical.}

%\todo[inline]{Knox: didn't you derive a more extensive X-ray source catalog yourself?  Should we be comparing to that?  (WPB)}

\subsection{Historical Supernovae in \gal}
In our 2011 WIYN emission-line images, we also searched for emission at the positions of all of the nine historical SNe that had occurred in \gal. at the time.  We detected emission from only two of these : SN\,1980K and SN\,2004et.  Both would probably have been among our \sii-selected SNR candidates, and SN\,1980K would also have attracted notice because of its relatively high \oiii:\ha\ ratio as well; however, we noted these in an explicit search of the positions of all the historical SNe in \gal.  Thus we have not included them in Tables \ref{table_candidates} or \ref{table_snr_spec}.
We  obtained GMOS spectra of both, as shown in Fig.~\ref{historical_SNe}.  Unlike the spectra from the other SNRs in our sample, the lines from both of these are highly velocity-broadened, the signature of fast shock waves in these young SN-SNR transition objects and fast-moving ejecta.

The Type IIL SN\,1980K has been frequently observed over the years since its explosion, and its transition from late-time SN to a developing SNR has been monitored both photometrically and spectroscopically \citep[e.g.,][]{uomoto86, fesen90, fesen94, fesen99b, milisavljevic12}.  The 2014 GMOS spectrum, taken 3 December 2014, about  34 yr past maximum light, is qualitatively similar to the 30 yr spectrum shown by \citet{milisavljevic12}, with broad, asymmetric lines from \ha, \oi, \oiii, and a feature near 7100 \AA\ that is probably [\ion{Fe}{2}]\,$\lambda$\,7155, possibly blended with [\ion{Ar}{3}]\,$\lambda$7136, all appearing above a faint, blue continuum \citep[see][for a discussion of the 7100 \AA\ feature]{fesen94}.  It appears that broad \SiiL\ with a similar asymmetric profile may also be present, blended with the red side of the  \ha\ line.  We estimate that the broad \ha\ flux is $8.4 \pm 1.0 \times 10^{-16} \FLUX$, slightly lower than that of $10 \pm 2 \times 10^{-16} \FLUX$ reported by \citet{milisavljevic12} for their spectrum taken in October 1010, just over four years earlier.  This continues the gradual decline they noted from that of $13 \pm 2 \times 10^{-16} \FLUX$ measured by \citet{fesen99b} in November 1997, which in turn represented a fading of $\sim 25\%$ from the levels observed in the early 1990s. A similar fading of broad lines with time has been observed in M83 for SN 1957D \citep{long12}.

SN\,2004et, classified as Type IIP, was also well observed early on and as it made the transition to its nebular stage \citep[e.g.,][]{sahu06, maguire10, jerkstrand12}; however, there appear to be no published spectra from later than 1.5 yr past its explosion.  The spectrum in Fig.~\ref{historical_SNe} shows its recovery at an age of just over 10 years.  The optical spectrum is dominated by broad, asymmetric lines of \ha, \oi, and a feature at $\sim$7150 - 7400 \AA\ that is probably a blend of (primarily)  [\ion{Fe}{2}] $\lambda 7155$ and [\ion{Ca}{2}] $\lambda\lambda 7291, 7324$, features that were prominent in its late nebular spectra \citep{sahu06, maguire10}.  (Unfortunately, this emission extends past the end of our spectrum, making it harder to identify.)  As in the case of SN\,1980K, it is possible that faint, broad \SiiL\ lines may be present, blended with the red side of \ha.  The \ha\ line flux is difficult to measure because it is so broad as to to be blended with \oi\ and, possibly, \sii; furthermore, the continuum level is also uncertain.  Our best flux estimate for \ha\ is $7 \pm 2 \times 10^{-16} \FLUX$.

%has had fewer late-time observations than SN\,1980K, perhaps due in part to the burgeoning number of SN discoveries over the past two decades.  

% Bill added the following paragraph, but feel free to edit as necessary.
The spectra of both SN\,1980K and SN\,2004et are quite  similar to the object B12-174a identified in our similar survey of M83 \citep{blair15}. The main difference is that for B12-174a the SN was not observed, even though its inferred age is $<$100 years.  All these objects form a transitional class between ``old SNe" and mature SNRs.  SN1957D in M83 also shows broad lines, but only for oxygen, and the line intensities have dropped significantly over $\sim$40 years \citep[][and references therein]{long12}.  These differences may be due to differing progenitor types, differing local ISM conditions, or both.  \citep[The \oiii\ lines would appear relatively stronger in both  SN\,1980K and SN\,2004et if these were dereddened, with $E(B-V) = 0.41$,][but even so they would not be nearly so O-dominated as SN\,1957D.]{fesen99b, sahu06} Since there are so few objects in this transitional class, these objects bear watching for temporal changes that should happen on observable time scales.  Such observations could illuminate this poorly understood phase of SNR evolution.

The fact that none of the other seven historical SNe in \gal, ranging in age from 3 to 94 years, were detected is noteworthy.  All those with well-determined SN classifications are ones that result from core-collapse SNe, and thus should have produced several $M_{\sun}$ of high-velocity ejecta---the scenario responsible for ejecta-dominated SNRs like Cas A, or SN\,1957D in M83.  Furthermore, since \gal\ is such a champion producer of SNe, it is reasonable to expect the remnants from dozens of core-collapse SNe younger than 1000 yr to be located there.  The fact that so few are detected as broad-line, ejecta-dominated remnants is also noteworthy.   This is similar to the case of M83, host to six historical SNe (or seven including B12-174a) and hence, also expected to have far more young SNRs than are detected.  \citet{winkler17} concluded that many of the SNRs are likely expanding into high-density environments, where remnants evolve rapidly to the point that they are dominated by swept-up material rather than by ejecta. At the other extreme, other SNe may have exploded in regions where earlier SNRs have evacuated the surrounding region, resulting in extremely faint SNRs.  It would seem that the situation is similar in \gal.

\section{Summary \label{sec:summary}}

We have carried out a new optical search for SNRs in \gal\ using interference filters to identify emission nebulae that have elevated \sii:\ha\ ratios compared to \hii\ regions.  Our search picked up all of the SNRs that had been identified by MF97, except one that was outside the region imaged here. Of the 147 SNR candidates we identified, we were able to obtain spectra of 102, and spectral confirmation of 89 objects.  There are 45 candidates  without spectra and 17 candidates with spectra which have spectroscopic \sii:\ha\ ratios less than the canonical ratio of 0.4 for regarding an emission nebula as a confirmed optical SNR.  Given the uncertainties, we have chosen to retain all of the 147 objects as candidates for being SNRs, though clearly only those with high ratios can be regarded as confirmed.  

Very few of the candidates are detected as SNRs at other wavelengths.  Only 7 are among the radio SNRs identified by \cite{lacey01}, most likely due to the limited sensitivity of the radio survey. Similarly only eight candidates have X-ray counterparts, which we attribute to a combination of higher absorption along the line of sight to \gal\ compared to some other galaxies at comparable distances, e.g. M83, and to the %somewhat 
lower exposure times for the {\em Chandra} study of \gal\ than for these other galaxies. 

\todo[inline]{Need at least one sentence on the historical SNe.  Frank- I have taken a crack at this}

We also inspected our images for evidence of emission at the sites of historical SNe in NGC6946, and obtained spectra of the two, SN1980K and SN2004ET, for which emission was apparent.  Both show the broad, and asymmetric lines that are typical of very young SNRs.  Although SN1980K is well-known as one of an unusual group of SN that continue to be observable long after the explosion, the last (published) spectrum  of SN2004ET was taken only 1.5 years after the SN explosion.  Our spectrum indicates that this object is still interacting with circumstellar material 10 years after the explosion.

Much more work is needed to fully characterize the SNR population of \gal, some of which we are currently working on.  These include HST studies in the optical to measure diameters and identify additional small diameter objects in crowded regions, infrared \feii\ 1.64 $\mu$m studies to identify SNRs in dusty regions or buried in complex \ha\ emission, and deeper radio studies to find and characterize the radio counterparts of the optical SNR population. 

\acknowledgments

Our WIYN images were obtained at Kitt Peak National Observatory of the National Optical Astronomy Observatories (NOAO Prop. ID 11A-0110; PI: Winkler), which is operated by the Association of Universities for Research in Astronomy (AURA) under a cooperative agreement with the National Science Foundation. The spectra were obtained at the Gemini Observatory (Gemini Prop. IDs GN-2014A-Q-84,GN-2014B-Q-83, GN-2015B-Q-91; PI: Winkler), which is operated by the Association of Universities for Research in Astronomy, Inc., under a cooperative agreement with the NSF on behalf of the Gemini partnership: the National Science Foundation (United States), National Research Council (Canada), CONICYT (Chile), Ministerio de Ciencia, Tecnolog\'{i}a e Innovaci\'{o}n Productiva (Argentina), Minist\'{e}rio da Ci\^{e}ncia, Tecnologia e Inova\c{c}\~{a}o (Brazil), and Korea Astronomy and Space Science Institute (Republic of Korea).  Partial support for the analysis of the data was provided by NASA through grant number HST-GO-14638 from the Space Telescope Science Institute, which is operated by AURA, Inc., under NASA contract NAS 5-26555. PFW acknowledges additional support from the NSF through grant AST-1714281.  WPB acknowledges partial support from the JHU Center for Astrophysical Sciences.

\vspace{5mm}
\facilities{NOAO:WIYN, Gemini:GMOS}

%% Similar to \facility{}, there is the optional \software command to allow 
%% authors a place to specify which programs were used during the creation of 
%% the manuscript. Authors should list each code and include either a
%% citation or url to the code inside ()s when available.

\software{astropy \citep{astropy} 
%          Cloudy \citep{2013RMxAA..49..137F}, 
 %         SExtractor \citep{1996A&AS..117..393B}
          }


\pagebreak

\bibliographystyle{aasjournal}

\bibliography{snr}

\clearpage

%\todo[inline]{(Note: I moved what was Table 2 to be Table 3 so the order agrees with the text presentation order.  WPB)}

\begin{deluxetable}{ccrr}
%\tabletypesize{\scriptsize}
%\rotate
%\tablewidth{600pt}
\tablewidth{0pt}
\tablecaption{WIYN Imaging Observations of NGC\,6946}

\tablehead{
\colhead {} & \multicolumn{2}{c}{Filter} & \colhead {}\\ 
\cline{2-3}  
\colhead{Designation} &
%\colhead{Designation} &
\colhead{$\rm \lambda_{c}$(\AA)} &
%\colhead{$\rm \lambda_{c}[corr]$(\AA)\tablenotemark{b}} &
\colhead{$\Delta \lambda$(\AA)\tablenotemark{a}} &
\colhead {Exposure (s)} 
}

\startdata
\oiii &  5010 &60\phn\phn & $11\times800$ \phn\phn  \\
Green Continuum  & 5127& 100\phn\phn&$11\times500$ \phn\phn \\
H$\alpha$  & 6563 & 27\phn\phn    & $10\times800$ \phn\phn \\
\sii\tablenotemark{b} & 6723 & 63\phn\phn & $10\times800$ \phn\phn  \\
Red Continuum  & 6840& 93\phn\phn &$10\times600$ \phn\phn \\
\enddata

%\tablenotetext{a}{Nominal central wavelength.}
\tablenotetext{a}{Full width at half maximum in the WIYN f/6.3 beam.}
\tablenotetext{b}{WIYN Observatory filter W037; other filters are PFW custom.}

\label{imaging_obsns}
\end{deluxetable}



\startlongtable

\clearpage




%\begin{center}
\begin{deluxetable}{rrccrrrcc}
\tabletypesize{\scriptsize}

\rotate\tablecaption{SNR Candidates in NGC~6946 }
\tablehead{\colhead{Source} & 
 \colhead{Other~Names$^b$} & 
 \colhead{R.A.} & 
 \colhead{Decl.} & 
 \colhead{GCD} & 
 \colhead{H$\alpha$~Flux$^a$} & 
 \colhead{\sii:H$\alpha$} & 
 \colhead{Spectrum} & 
 \colhead{Confirmed} 
\\
\colhead{~} & 
 \colhead{~} & 
 \colhead{(2000)} & 
 \colhead{(2000)} & 
 \colhead{(kpc)} & 
 \colhead{~} & 
% \colhead{(10$^{-17}$ergs~cm$^{-2}$s$^{-1}$)} & 
 \colhead{~} & 
 \colhead{~} & 
 \colhead{~} 
 \\
 \colhead{(1)} &
 \colhead{(2)} &
 \colhead{(3)} &
 \colhead{(4)} &
 \colhead{(5)} &
 \colhead{(6)} &
 \colhead{(7)} &
 \colhead{(8)} &
 \colhead{(9)}  
}
%\tabletypesize{\scriptsize}
\tablewidth{0pt}\startdata
L19-001 &  -- &  20:34:14.999 &  60:10:44.28 &  10.4 &  52.4 &  0.21 &  05.18 &  no \\ 
L19-002 &  -- &  20:34:15.475 &  60:07:31.59 &  9.6 &  63.5 &  0.34 &  -- &  -- \\ 
L19-003 &  -- &  20:34:15.778 &  60:08:26.03 &  9.2 &  215.7 &  1.14 &  -- &  -- \\ 
L19-004 &  -- &  20:34:16.409 &  60:08:27.31 &  9.0 &  32.7 &  0.61 &  02.25 &  no \\ 
L19-005 &  -- &  20:34:16.680 &  60:07:30.81 &  9.3 &  119.6 &  0.42 &  08.17 &  no \\ 
L19-006 &  -- &  20:34:17.543 &  60:10:58.32 &  10.1 &  97.0 &  0.66 &  05.09 &  yes \\ 
L19-007 &  -- &  20:34:17.948 &  60:10:00.42 &  9.1 &  92.1 &  0.49 &  02.10 &  yes \\ 
L19-008 &  -- &  20:34:18.387 &  60:10:47.32 &  9.7 &  539.8 &  0.33 &  -- &  -- \\ 
L19-009 &  -- &  20:34:18.841 &  60:11:08.85 &  10.0 &  32.8 &  0.86 &  05.01 &  yes \\ 
L19-010 &  -- &  20:34:19.172 &  60:08:57.45 &  8.3 &  251.3 &  0.40 &  02.21 &  yes \\ 
L19-011 &  -- &  20:34:20.602 &  60:09:06.77 &  8.0 &  55.6 &  0.52 &  02.11 &  yes \\ 
L19-012 &  -- &  20:34:21.958 &  60:08:57.79 &  7.6 &  90.1 &  0.50 &  -- &  -- \\ 
L19-013 &  -- &  20:34:22.695 &  60:06:13.39 &  9.4 &  15.4 &  0.82 &  08.01 &  yes \\ 
L19-014 &  MF-001; &  20:34:23.381 &  60:08:18.70 &  7.3 &  95.7 &  0.62 &  02.01 &  yes \\ 
L19-015 &  -- &  20:34:23.393 &  60:11:35.31 &  9.6 &  17.0 &  0.87 &  05.19 &  yes \\ 
L19-016 &  -- &  20:34:24.427 &  60:11:25.82 &  9.1 &  168.7 &  0.41 &  05.10 &  yes \\ 
L19-017 &  -- &  20:34:24.932 &  60:09:46.47 &  7.2 &  285.9 &  0.31 &  02.22 &  no \\ 
L19-018 &  -- &  20:34:25.373 &  60:08:56.43 &  6.7 &  64.9 &  0.39 &  -- &  -- \\ 
L19-019 &  MF-002; &  20:34:26.000 &  60:11:10.50 &  8.4 &  102.6 &  0.65 &  05.02 &  yes \\ 
L19-020 &  -- &  20:34:26.062 &  60:13:22.84 &  12.2 &  16.9 &  0.60 &  -- &  -- \\ 
L19-021 &  -- &  20:34:26.173 &  60:10:11.85 &  7.2 &  93.5 &  0.41 &  -- &  -- \\ 
L19-022 &  -- &  20:34:27.648 &  60:11:12.17 &  8.1 &  45.0 &  0.60 &  -- &  -- \\ 
L19-023 &  -- &  20:34:28.219 &  60:11:37.96 &  8.7 &  4.9 &  1.77 &  -- &  -- \\ 
L19-024 &  -- &  20:34:28.320 &  60:13:21.96 &  11.8 &  53.3 &  0.77 &  -- &  -- \\ 
L19-025 &  -- &  20:34:28.328 &  60:07:04.17 &  7.2 &  20.1 &  0.96 &  08.02 &  yes \\ 
L19-026 &  -- &  20:34:28.395 &  60:08:09.46 &  6.2 &  34.8 &  0.61 &  -- &  -- \\ 
L19-027 &  -- &  20:34:28.443 &  60:07:33.36 &  6.7 &  17.0 &  0.63 &  -- &  -- \\ 
L19-028 &  -- &  20:34:28.856 &  60:07:45.37 &  6.4 &  214.5 &  0.34 &  02.18 &  no \\ 
L19-029 &  F08-08; &  20:34:29.170 &  60:10:51.08 &  7.3 &  11.6 &  1.26 &  -- &  -- \\ 
L19-030 &  -- &  20:34:30.127 &  60:10:24.41 &  6.5 &  8.6 &  0.71 &  05.20 &  yes \\ 
L19-031 &  -- &  20:34:31.672 &  60:10:28.00 &  6.2 &  78.0 &  0.65 &  05.05 &  yes \\ 
L19-032 &  -- &  20:34:32.604 &  60:10:27.92 &  6.0 &  81.1 &  0.47 &  05.05 &  no \\ 
L19-033 &  -- &  20:34:33.050 &  60:11:25.71 &  7.4 &  133.6 &  0.49 &  05.11 &  yes \\ 
L19-034 &  -- &  20:34:33.307 &  60:09:46.72 &  5.1 &  13.0 &  1.12 &  -- &  -- \\ 
L19-035 &  MF-003; &  20:34:33.654 &  60:09:52.00 &  5.1 &  13.8 &  1.32 &  -- &  -- \\ 
L19-036 &  MF-004; &  20:34:33.853 &  60:09:25.00 &  4.7 &  81.4 &  0.97 &  02.02 &  yes \\ 
L19-037 &  -- &  20:34:36.631 &  60:11:34.35 &  7.0 &  186.0 &  0.44 &  05.03 &  yes \\ 
L19-038 &  -- &  20:34:37.375 &  60:07:15.04 &  5.4 &  42.2 &  0.66 &  02.03 &  yes \\ 
L19-039 &  -- &  20:34:37.436 &  60:11:31.36 &  6.8 &  36.1 &  0.77 &  04.01 &  yes \\ 
L19-040 &  MF-005; &  20:34:37.761 &  60:08:52.60 &  3.6 &  23.5 &  0.90 &  08.07 &  yes \\ 
L19-041 &  MF-006; &  20:34:37.811 &  60:11:54.42 &  7.4 &  36.7 &  0.91 &  05.04 &  yes \\ 
L19-042 &  MF-007; &  20:34:37.980 &  60:07:22.30 &  5.1 &  17.5 &  1.31 &  02.04 &  yes \\ 
L19-043 &  -- &  20:34:38.359 &  60:06:09.44 &  7.3 &  130.2 &  0.47 &  -- &  -- \\ 
L19-044 &  -- &  20:34:38.901 &  60:06:57.66 &  5.7 &  80.6 &  0.53 &  08.08 &  yes \\ 
L19-045 &  -- &  20:34:39.146 &  60:09:18.99 &  3.3 &  405.4 &  0.32 &  -- &  -- \\ 
L19-046 &  -- &  20:34:39.186 &  60:08:13.87 &  3.7 &  43.8 &  0.58 &  02.05 &  yes \\ 
L19-047 &  -- &  20:34:39.648 &  60:07:25.96 &  4.8 &  1.6 &  2.50 &  -- &  -- \\ 
L19-048 &  -- &  20:34:40.627 &  60:06:53.47 &  5.7 &  79.8 &  0.40 &  08.09 &  yes \\ 
L19-049 &  -- &  20:34:40.728 &  60:08:33.96 &  3.1 &  46.0 &  0.53 &  02.23 &  yes \\ 
L19-050 &  -- &  20:34:41.019 &  60:05:57.87 &  7.5 &  11.7 &  1.02 &  -- &  -- \\ 
L19-051 &  -- &  20:34:41.315 &  60:11:12.97 &  5.5 &  23.0 &  0.73 &  04.21 &  yes \\ 
L19-052 &  -- &  20:34:41.318 &  60:04:54.88 &  9.7 &  81.6 &  0.43 &  -- &  -- \\ 
L19-053 &  -- &  20:34:41.534 &  60:11:29.95 &  6.1 &  66.8 &  0.47 &  05.21 &  yes \\ 
L19-054 &  -- &  20:34:41.931 &  60:05:49.76 &  7.8 &  102.9 &  0.44 &  08.03 &  yes \\ 
L19-055 &  -- &  20:34:42.435 &  60:09:15.98 &  2.5 &  5.8 &  1.87 &  02.13 &  yes \\ 
L19-056 &  -- &  20:34:43.081 &  60:11:39.38 &  6.2 &  81.6 &  0.40 &  04.11 &  no \\ 
L19-057 &  -- &  20:34:43.323 &  60:10:11.07 &  3.3 &  186.7 &  0.44 &  -- &  -- \\ 
L19-058 &  -- &  20:34:43.527 &  60:07:51.69 &  3.5 &  24.7 &  0.67 &  -- &  -- \\ 
L19-059 &  MF-008; &  20:34:43.967 &  60:08:24.40 &  2.6 &  54.2 &  0.62 &  02.14 &  yes \\ 
L19-060 &  -- &  20:34:44.605 &  60:08:17.33 &  2.7 &  62.9 &  0.37 &  02.15 &  yes \\ 
L19-061 &  -- &  20:34:45.130 &  60:12:36.38 &  8.0 &  8.6 &  1.31 &  04.12 &  yes \\ 
L19-062 &  -- &  20:34:45.673 &  60:07:21.18 &  4.3 &  196.3 &  0.35 &  02.24 &  yes \\ 
L19-063 &  -- &  20:34:46.915 &  60:12:19.39 &  7.2 &  34.7 &  0.68 &  04.22 &  yes \\ 
L19-064 &  -- &  20:34:47.188 &  60:08:20.24 &  2.2 &  78.7 &  0.48 &  08.10 &  yes \\ 
L19-065 &  -- &  20:34:47.367 &  60:08:22.70 &  2.1 &  109.3 &  0.63 &  02.09 &  yes \\ 
L19-066 &  L97-34; &  20:34:47.753 &  60:09:58.69 &  2.1 &  56.7 &  0.79 &  04.13 &  yes \\ 
L19-067 &  -- &  20:34:48.091 &  60:07:50.49 &  3.2 &  97.0 &  0.44 &  08.11 &  yes \\ 
L19-068 &  -- &  20:34:48.638 &  60:09:24.39 &  1.0 &  159.4 &  0.44 &  07.01 &  yes \\ 
L19-069 &  -- &  20:34:48.715 &  60:08:23.44 &  2.0 &  137.9 &  0.41 &  01.01 &  yes \\ 
L19-070 &  -- &  20:34:49.657 &  60:07:37.00 &  3.6 &  59.1 &  0.50 &  03.10 &  yes \\ 
L19-071 &  -- &  20:34:49.795 &  60:09:41.27 &  1.2 &  69.4 &  0.40 &  -- &  -- \\ 
L19-072 &  -- &  20:34:49.952 &  60:07:53.49 &  3.0 &  50.1 &  0.54 &  06.10 &  yes \\ 
L19-073 &  -- &  20:34:50.023 &  60:09:43.27 &  1.3 &  86.1 &  0.49 &  -- &  -- \\ 
L19-074 &  -- &  20:34:50.358 &  60:09:45.17 &  1.3 &  79.0 &  0.38 &  02.16 &  yes \\ 
L19-075 &  -- &  20:34:50.371 &  60:09:51.77 &  1.5 &  579.2 &  0.24 &  -- &  -- \\ 
L19-076 &  F08-43;B14-20; &  20:34:50.798 &  60:07:48.35 &  3.2 &  159.0 &  0.31 &  03.11 &  yes \\ 
L19-077 &  L97-48;F08-45; &  20:34:50.940 &  60:10:20.91 &  2.6 &  3662.0 &  0.29 &  -- &  -- \\ 
L19-078 &  -- &  20:34:51.292 &  60:05:20.47 &  8.7 &  227.2 &  0.44 &  -- &  -- \\ 
L19-079 &  MF-009;L97-51; &  20:34:51.453 &  60:07:39.30 &  3.5 &  116.0 &  0.62 &  07.11 &  yes \\ 
L19-080 &  MF-010;F08-47; &  20:34:51.567 &  60:09:09.20 &  0.2 &  78.5 &  0.74 &  02.06 &  yes \\ 
L19-081 &  -- &  20:34:51.657 &  60:09:57.17 &  1.6 &  86.3 &  0.47 &  01.02 &  no \\ 
L19-082 &  MF-011; &  20:34:52.473 &  60:07:28.20 &  4.0 &  39.1 &  0.89 &  03.12 &  yes \\ 
L19-083 &  -- &  20:34:52.510 &  60:10:01.89 &  1.8 &  68.8 &  0.70 &  02.07 &  yes \\ 
L19-084 &  -- &  20:34:52.563 &  60:10:52.29 &  3.7 &  187.2 &  0.47 &  04.15 &  yes \\ 
L19-085 &  -- &  20:34:53.093 &  60:08:14.09 &  2.3 &  9.8 &  1.24 &  07.24 &  yes \\ 
L19-086 &  L97-68; &  20:34:53.712 &  60:07:13.89 &  4.6 &  85.9 &  0.64 &  02.08 &  yes \\ 
L19-087 &  MF-012; &  20:34:54.308 &  60:11:03.40 &  4.0 &  33.4 &  0.96 &  04.02 &  yes \\ 
L19-088 &  -- &  20:34:54.414 &  60:10:55.89 &  3.8 &  10.1 &  1.38 &  01.03 &  yes \\ 
L19-089 &  -- &  20:34:54.553 &  60:05:08.59 &  9.3 &  178.4 &  0.63 &  08.16 &  yes \\ 
L19-090 &  -- &  20:34:54.802 &  60:10:06.79 &  2.0 &  12.3 &  1.25 &  02.17 &  yes \\ 
L19-091 &  -- &  20:34:54.868 &  60:10:34.59 &  3.0 &  55.7 &  0.64 &  07.12 &  yes \\ 
L19-092 &  -- &  20:34:55.621 &  60:11:13.71 &  4.4 &  43.2 &  0.51 &  -- &  -- \\ 
L19-093 &  MF-013; &  20:34:55.900 &  60:07:49.20 &  3.5 &  141.5 &  0.50 &  03.02 &  yes \\ 
L19-094 &  F08-53; &  20:34:56.579 &  60:08:19.91 &  2.5 &  87.5 &  0.76 &  01.04 &  yes \\ 
L19-095 &  MF-014;B14-25; &  20:34:57.813 &  60:08:10.10 &  3.0 &  71.4 &  0.71 &  01.05 &  yes \\ 
L19-096 &  -- &  20:34:58.491 &  60:08:01.78 &  3.3 &  9.2 &  1.39 &  07.13 &  yes \\ 
L19-097 &  MF-015; &  20:35:00.314 &  60:11:46.00 &  5.8 &  200.9 &  0.62 &  04.03 &  yes \\ 
L19-098 &  MF-016;L97-85;F08-63;B14-29; &  20:35:00.721 &  60:11:30.90 &  5.3 &  1184.0 &  0.95 &  01.06 &  yes \\ 
L19-099 &  MF-017; &  20:35:01.154 &  60:12:00.10 &  6.3 &  44.4 &  0.57 &  04.04 &  yes \\ 
L19-100 &  -- &  20:35:02.244 &  60:11:05.17 &  4.6 &  274.1 &  0.48 &  01.07 &  yes \\ 
L19-101 &  MF-018; &  20:35:02.380 &  60:06:31.50 &  7.0 &  187.7 &  0.57 &  03.04 &  yes \\ 
L19-102 &  -- &  20:35:02.930 &  60:11:27.18 &  5.3 &  59.8 &  0.51 &  06.14 &  yes \\ 
L19-103 &  -- &  20:35:03.167 &  60:10:41.88 &  4.0 &  25.3 &  0.87 &  01.08 &  yes \\ 
L19-104 &  MF-019; &  20:35:03.302 &  60:05:28.80 &  9.3 &  65.9 &  0.71 &  03.13 &  yes \\ 
L19-105 &  -- &  20:35:03.586 &  60:06:23.35 &  7.4 &  78.5 &  0.41 &  -- &  -- \\ 
L19-106 &  -- &  20:35:04.056 &  60:11:15.57 &  5.1 &  14.1 &  1.33 &  04.16 &  yes \\ 
L19-107 &  -- &  20:35:04.191 &  60:11:18.47 &  5.2 &  48.3 &  0.72 &  -- &  -- \\ 
L19-108 &  L97-88; &  20:35:04.222 &  60:09:53.52 &  3.2 &  43.0 &  0.64 &  06.15 &  no \\ 
L19-109 &  -- &  20:35:04.269 &  60:06:52.07 &  6.5 &  10.9 &  1.13 &  03.14 &  yes \\ 
L19-110 &  -- &  20:35:04.996 &  60:05:32.94 &  9.3 &  47.9 &  0.50 &  -- &  -- \\ 
L19-111 &  MF-020; &  20:35:05.634 &  60:10:00.80 &  3.6 &  47.4 &  0.60 &  06.16 &  yes \\ 
L19-112 &  L97-95; &  20:35:05.688 &  60:11:07.64 &  5.1 &  382.9 &  0.32 &  04.06 &  yes \\ 
L19-113 &  -- &  20:35:06.886 &  60:07:58.35 &  5.0 &  40.1 &  0.59 &  03.15 &  yes \\ 
L19-114 &  -- &  20:35:06.964 &  60:09:57.03 &  3.9 &  97.6 &  0.51 &  04.07 &  yes \\ 
L19-115 &  -- &  20:35:07.074 &  60:05:57.33 &  8.8 &  246.7 &  0.36 &  -- &  -- \\ 
L19-116 &  MF-021; &  20:35:08.800 &  60:06:03.00 &  8.8 &  63.1 &  0.45 &  -- &  -- \\ 
L19-117 &  -- &  20:35:08.886 &  60:10:13.03 &  4.5 &  8.8 &  1.08 &  -- &  -- \\ 
L19-118 &  -- &  20:35:09.562 &  60:09:13.13 &  4.4 &  238.9 &  0.39 &  06.17 &  no \\ 
L19-119 &  MF-022; &  20:35:09.613 &  60:12:30.00 &  8.0 &  123.6 &  0.73 &  -- &  -- \\ 
L19-120 &  -- &  20:35:09.870 &  60:06:13.32 &  8.6 &  19.1 &  0.82 &  -- &  -- \\ 
L19-121 &  -- &  20:35:10.223 &  60:06:26.70 &  8.3 &  82.6 &  0.49 &  03.16 &  yes \\ 
L19-122 &  -- &  20:35:10.542 &  60:06:41.32 &  7.9 &  23.1 &  0.78 &  -- &  -- \\ 
L19-123 &  -- &  20:35:10.634 &  60:10:40.92 &  5.3 &  189.1 &  0.39 &  01.11 &  yes \\ 
L19-124 &  F08-74; &  20:35:10.894 &  60:08:56.89 &  4.9 &  824.5 &  0.33 &  06.04 &  no \\ 
L19-125 &  -- &  20:35:11.040 &  60:08:27.11 &  5.3 &  69.7 &  0.49 &  01.12 &  yes \\ 
L19-126 &  -- &  20:35:11.451 &  60:11:11.94 &  6.1 &  109.9 &  0.57 &  04.17 &  yes \\ 
L19-127 &  MF-023; &  20:35:11.600 &  60:07:41.20 &  6.4 &  183.4 &  0.51 &  03.05 &  yes \\ 
L19-128 &  -- &  20:35:11.896 &  60:09:28.60 &  5.0 &  20.4 &  0.88 &  06.18 &  yes \\ 
L19-129 &  -- &  20:35:11.942 &  60:04:03.68 &  13.3 &  341.9 &  0.34 &  -- &  -- \\ 
L19-130 &  -- &  20:35:12.254 &  60:06:37.60 &  8.3 &  69.0 &  0.57 &  -- &  -- \\ 
L19-131 &  -- &  20:35:12.615 &  60:09:09.71 &  5.2 &  59.5 &  0.62 &  01.13 &  yes \\ 
L19-132 &  -- &  20:35:13.615 &  60:08:58.91 &  5.5 &  110.8 &  0.54 &  07.25 &  yes \\ 
L19-133 &  -- &  20:35:14.443 &  60:07:12.69 &  7.7 &  8.5 &  1.10 &  07.18 &  yes \\ 
L19-134 &  -- &  20:35:16.524 &  60:07:50.07 &  7.3 &  11.0 &  0.77 &  -- &  -- \\ 
L19-135 &  MF-024; &  20:35:16.927 &  60:11:05.40 &  7.0 &  55.9 &  0.86 &  01.15 &  yes \\ 
L19-136 &  -- &  20:35:17.333 &  60:10:27.34 &  6.6 &  20.5 &  0.80 &  04.18 &  yes \\ 
L19-137 &  -- &  20:35:17.556 &  60:07:19.29 &  8.2 &  199.7 &  0.47 &  -- &  -- \\ 
L19-138 &  F08-82; &  20:35:20.028 &  60:09:33.92 &  7.0 &  87.8 &  0.61 &  06.05 &  yes \\ 
L19-139 &  -- &  20:35:20.796 &  60:09:52.71 &  7.2 &  15.6 &  1.17 &  -- &  -- \\ 
L19-140 &  MF-025; &  20:35:21.114 &  60:08:44.10 &  7.6 &  202.7 &  0.63 &  06.06 &  yes \\ 
L19-141 &  -- &  20:35:23.024 &  60:08:21.17 &  8.3 &  199.7 &  0.38 &  01.18 &  yes \\ 
L19-142 &  -- &  20:35:23.662 &  60:08:47.66 &  8.2 &  129.2 &  0.44 &  07.20 &  no \\ 
L19-143 &  -- &  20:35:24.219 &  60:07:42.45 &  9.2 &  123.7 &  0.41 &  03.17 &  no \\ 
L19-144 &  -- &  20:35:24.660 &  60:06:57.15 &  10.3 &  18.1 &  0.80 &  -- &  -- \\ 
L19-145 &  -- &  20:35:25.243 &  60:07:26.90 &  9.8 &  298.6 &  0.38 &  -- &  -- \\ 
L19-146 &  MF-026; &  20:35:25.513 &  60:07:51.30 &  9.4 &  57.2 &  0.67 &  -- &  -- \\ 
L19-147 &  MF-027; &  20:35:26.112 &  60:08:43.00 &  8.8 &  202.3 &  0.65 &  01.19 &  yes \\ 
\enddata 
\tablenotetext{a}{(10$^{-17}$ergs~cm$^{-2}$s$^{-1}$)}
\tablenotetext{b}{MF = \cite{matonick97}; B08 = \cite{boomsma08};  B14 = \cite{bruursema14}; F08 = \cite{fridriksson08}; L97 = ??}
\label{table_candidates}
\end{deluxetable}
%\end{center}


\pagebreak

%  SN1006 core sample paper,  Table 1
\begin{deluxetable}{ccc}
%\tabletypesize{\scriptsize}
%\rotate
%\tablewidth{600pt}
\tablewidth{0pt}
\tablecaption{Gemini-N/GMOS Multi-Object Spectroscopy Observations of NGC\,6946}

\tablehead{
\colhead{Mask No.} &
\colhead{Date (UT)}  &
%\colhead{CWLs} &
\colhead{Total Exposure (s)\tablenotemark{a} }
%\colhead{Comments} 
%\colhead{$\rm \lambda_{c}[corr]$(\AA)\tablenotemark{b}} &
%\colhead{$\Delta \lambda$(\AA)\tablenotemark{a}} &
%\colhead {Exposure (s)} 
}

\startdata
%2 & 7 Apr 2011 & $7 \times 1000$ &  Variable clouds and seeing  \\[-8pt]
% &  & $7 \times 1000$ &    \\[-9pt]
 %& 7 Apr 2011 &  &  Variable clouds and seeing  \\[-9pt]
%2 &    &  $2 \times 1500$ &  \\[-1pt]
%&  9 Apr 2011  & $2 \times 1400$ & Clear; seeing $<1\arcsec$  \\[2pt]
%% & 9 Apr 2011 & $ 1500 + 1350 $ &  Clear; seeing $\lesssim 0\farcs 7$  \\
 1 & 30 Jul 2014 & 3 CWLs $\times 2 \times 1800$  \\[2pt]
 2 & 24 Sep 2014 & 3 CWLs $\times 2 \times 1800$ \\[2pt]
3 & 25-30 Sep 2014 & 3 CWLs $\times 2 \times 1800$  \\[2pt]
4 & 26 Oct, 19 Nov 2014 & 3 CWLs $\times 2 \times 1800$  \\[2pt]
5 & 21-27 Nov 2014 & 3 CWLs $\times 2 \times 1800$  \\[2pt]
6\tablenotemark{b}  & 14-17 Dec 2014 & 2 CWLs $\times 2 \times 1800$  \\[2pt]
7 & 14 Sep 2015 & 3 CWLs $\times 3 \times 1200$  \\[2pt]
8 & 20 Sep\,-\,19 Oct 2015 & 4 CWLs $\times 3 \times 1000$  \\[2pt]
\enddata

\tablenotetext{a}{Number of different Central Wavelength Settings $\times$ number of exposures at each CWL $\times$ individual exposure time.}
\tablenotetext{b}{Observations for mask 6, done late in the 2014B semester, were incomplete.  Many of the same objects were re-observed with mask 7 in 2015B.}


\label{obs_log}
\end{deluxetable}




\clearpage

\startlongtable

%\begin{center}
\begin{deluxetable}{rrrrrrrrrrr}
% \rotate
\tablecaption{Emission line fluxes of SNR candidates$^{a,b,c}$ }
\tablehead{\colhead{Source} & 
 \colhead{H$\alpha$~flux} & 
 \colhead{H$\beta$} & 
 \colhead{[0III]$\lambda$5007} & 
 \colhead{[OI]$\lambda$6300} & 
 \colhead{H$\alpha$} & 
 \colhead{[NII]$\lambda$6584} & 
 \colhead{[SII]$\lambda$6717} & 
 \colhead{[SII]$\lambda$6731} & 
 \colhead{[SII]:H$\alpha$} & 
 \colhead{FWHM} 
}
\tabletypesize{\scriptsize}
\tablewidth{0pt}\startdata
L19-000 &  183 &  76 &  -- &  -- &  300 &  38 &  33 &  30 &  0.21 &  6.8 \\ 
L19-003 &  68 &  53 &  -- &  -- &  300 &  $\sim$93 &  69 &  47 &  0.39 &  7.1 \\ 
L19-004 &  216 &  80 &  $\sim$182 &  -- &  300 &  61 &  65 &  43 &  0.36 &  8.3 \\ 
L19-005 &  181 &  79 &  -- &  -- &  300 &  102 &  170 &  113 &  0.94 &  8.4 \\ 
L19-006 &  123 &  54 &  87 &  -- &  300 &  149 &  84 &  58 &  0.47 &  7.4 \\ 
L19-008 &  122 &  29 &  81 &  40 &  300 &  114 &  180 &  138 &  1.06 &  7.6 \\ 
L19-009 &  57 &  $\sim$49 &  -- &  $\sim$38 &  300 &  $\sim$111 &  79 &  53 &  0.44 &  9.4 \\ 
L19-010 &  28 &  $\sim$42 &  $\sim$202 &  $\sim$18 &  300 &  $\sim$134 &  $\sim$102 &  $\sim$78 &  $\sim$0.60 &  8.5 \\ 
L19-012 &  19 &  $\sim$61 &  $\sim$165 &  $\sim$76 &  300 &  $\sim$148 &  $\sim$132 &  $\sim$127 &  $\sim$0.86 &  7.0 \\ 
L19-013 &  132 &  38 &  108 &  40 &  300 &  164 &  123 &  89 &  0.71 &  7.4 \\ 
L19-014 &  52 &  -- &  146 &  -- &  300 &  139 &  117 &  112 &  0.76 &  9.6 \\ 
L19-015 &  102 &  -- &  32 &  36 &  300 &  135 &  149 &  103 &  0.84 &  6.1 \\ 
L19-016 &  248 &  31 &  15 &  73 &  300 &  95 &  49 &  29 &  0.26 &  7.6 \\ 
L19-018 &  85 &  36 &  139 &  209 &  300 &  180 &  239 &  171 &  1.37 &  6.5 \\ 
L19-024 &  21 &  -- &  $\sim$166 &  $\sim$81 &  300 &  $\sim$225 &  182 &  133 &  1.05 &  7.8 \\ 
L19-027 &  150 &  29 &  9 &  -- &  300 &  99 &  57 &  43 &  0.33 &  5.5 \\ 
L19-029 &  31 &  -- &  -- &  874 &  300 &  $\sim$168 &  177 &  121 &  0.99 &  6.2 \\ 
L19-030 &  179 &  69 &  110 &  67 &  300 &  217 &  189 &  134 &  1.08 &  7.6 \\ 
L19-031 &  501 &  55 &  $\sim$13 &  $\sim$6 &  300 &  119 &  64 &  49 &  0.38 &  7.4 \\ 
L19-032 &  157 &  75 &  -- &  -- &  300 &  179 &  132 &  85 &  0.72 &  7.2 \\ 
L19-035 &  89 &  35 &  56 &  81 &  300 &  296 &  218 &  161 &  1.26 &  7.7 \\ 
L19-036 &  43 &  -- &  -- &  $\sim$55 &  300 &  $\sim$142 &  243 &  167 &  1.37 &  4.0 \\ 
L19-037 &  28 &  $\sim$56 &  $\sim$67 &  $\sim$63 &  300 &  $\sim$224 &  173 &  115 &  0.96 &  6.1 \\ 
L19-038 &  78 &  $\sim$31 &  $\sim$76 &  $\sim$-357 &  300 &  240 &  189 &  140 &  1.09 &  7.1 \\ 
L19-039 &  20 &  -- &  $\sim$23 &  $\sim$60 &  300 &  $\sim$314 &  231 &  166 &  1.32 &  7.0 \\ 
L19-040 &  55 &  117 &  $\sim$177 &  -- &  300 &  $\sim$202 &  156 &  146 &  1.01 &  5.9 \\ 
L19-041 &  26 &  $\sim$39 &  $\sim$158 &  $\sim$89 &  300 &  390 &  240 &  175 &  1.38 &  7.3 \\ 
L19-043 &  15 &  $\sim$88 &  $\sim$106 &  $\sim$49 &  300 &  $\sim$180 &  $\sim$117 &  $\sim$69 &  $\sim$0.62 &  6.7 \\ 
L19-045 &  31 &  $\sim$32 &  $\sim$92 &  $\sim$93 &  300 &  366 &  177 &  149 &  1.09 &  7.7 \\ 
L19-047 &  69 &  50 &  56 &  47 &  300 &  146 &  79 &  55 &  0.45 &  6.9 \\ 
L19-048 &  56 &  $\sim$35 &  $\sim$76 &  $\sim$13 &  300 &  185 &  95 &  65 &  0.53 &  7.5 \\ 
L19-050 &  45 &  -- &  182 &  242 &  300 &  303 &  191 &  147 &  1.13 &  7.5 \\ 
L19-052 &  239 &  64 &  -- &  20 &  300 &  103 &  84 &  63 &  0.49 &  6.5 \\ 
L19-053 &  88 &  56 &  -- &  -- &  300 &  107 &  79 &  56 &  0.45 &  6.7 \\ 
L19-054 &  9 &  $\sim$86 &  -- &  $\sim$65 &  300 &  $\sim$420 &  $\sim$227 &  $\sim$160 &  $\sim$1.29 &  6.6 \\ 
L19-055 &  97 &  44 &  $\sim$21 &  -- &  300 &  112 &  62 &  44 &  0.35 &  7.9 \\ 
L19-058 &  62 &  $\sim$40 &  -- &  -- &  300 &  264 &  126 &  90 &  0.72 &  7.2 \\ 
L19-059 &  50 &  $\sim$42 &  $\sim$46 &  $\sim$49 &  300 &  205 &  82 &  66 &  0.49 &  7.2 \\ 
L19-060 &  60 &  -- &  -- &  -- &  300 &  177 &  198 &  146 &  1.15 &  7.1 \\ 
L19-061 &  85 &  47 &  $\sim$51 &  $\sim$10 &  300 &  131 &  83 &  60 &  0.48 &  5.7 \\ 
L19-062 &  116 &  52 &  $\sim$26 &  -- &  300 &  118 &  85 &  69 &  0.51 &  8.4 \\ 
L19-063 &  56 &  $\sim$26 &  $\sim$39 &  -- &  300 &  $\sim$173 &  105 &  70 &  0.58 &  6.5 \\ 
L19-064 &  48 &  $\sim$8 &  $\sim$19 &  $\sim$60 &  300 &  232 &  170 &  137 &  1.02 &  6.1 \\ 
L19-065 &  36 &  -- &  -- &  -- &  300 &  350 &  314 &  198 &  1.71 &  5.9 \\ 
L19-066 &  69 &  55 &  -- &  -- &  300 &  157 &  118 &  93 &  0.70 &  7.0 \\ 
L19-067 &  92 &  $\sim$30 &  47 &  39 &  300 &  270 &  140 &  105 &  0.82 &  6.7 \\ 
L19-068 &  222 &  35 &  42 &  26 &  300 &  188 &  92 &  72 &  0.54 &  8.1 \\ 
L19-069 &  49 &  62 &  127 &  -- &  300 &  270 &  118 &  86 &  0.68 &  7.9 \\ 
L19-071 &  27 &  -- &  $\sim$77 &  $\sim$40 &  300 &  323 &  203 &  113 &  1.06 &  6.4 \\ 
L19-073 &  24 &  $\sim$32 &  -- &  258 &  300 &  739 &  368 &  302 &  2.24 &  8.1 \\ 
L19-075 &  47 &  -- &  78 &  104 &  300 &  607 &  132 &  182 &  1.05 &  9.2 \\ 
L19-078 &  22 &  $\sim$18 &  $\sim$62 &  $\sim$54 &  300 &  $\sim$343 &  $\sim$135 &  $\sim$97 &  $\sim$0.77 &  6.9 \\ 
L19-079 &  73 &  $\sim$13 &  158 &  78 &  300 &  637 &  185 &  178 &  1.21 &  9.5 \\ 
L19-080 &  99 &  $\sim$13 &  $\sim$47 &  $\sim$20 &  300 &  142 &  63 &  45 &  0.36 &  8.1 \\ 
L19-081 &  23 &  -- &  -- &  $\sim$111 &  300 &  $\sim$389 &  261 &  181 &  1.47 &  6.6 \\ 
L19-082 &  24 &  $\sim$32 &  $\sim$176 &  $\sim$87 &  300 &  629 &  234 &  182 &  1.39 &  7.7 \\ 
L19-083 &  119 &  51 &  24 &  41 &  300 &  140 &  109 &  76 &  0.62 &  7.8 \\ 
L19-084 &  37 &  -- &  $\sim$235 &  $\sim$43 &  300 &  493 &  182 &  41 &  0.74 &  8.6 \\ 
L19-085 &  28 &  -- &  $\sim$174 &  $\sim$58 &  300 &  393 &  245 &  128 &  1.24 &  7.5 \\ 
L19-086 &  100 &  42 &  46 &  121 &  300 &  263 &  204 &  156 &  1.20 &  7.1 \\ 
L19-087 &  24 &  $\sim$81 &  $\sim$271 &  $\sim$61 &  300 &  $\sim$232 &  216 &  157 &  1.24 &  5.4 \\ 
L19-088 &  11 &  -- &  -- &  -- &  300 &  $\sim$160 &  $\sim$138 &  $\sim$76 &  $\sim$0.71 &  11.0 \\ 
L19-089 &  27 &  -- &  -- &  -- &  300 &  499 &  180 &  155 &  1.12 &  7.6 \\ 
L19-090 &  9 &  $\sim$84 &  -- &  -- &  300 &  $\sim$387 &  $\sim$332 &  $\sim$253 &  $\sim$1.95 &  6.4 \\ 
L19-092 &  159 &  49 &  56 &  -- &  300 &  175 &  100 &  71 &  0.57 &  6.9 \\ 
L19-093 &  179 &  37 &  153 &  60 &  300 &  433 &  107 &  119 &  0.75 &  7.9 \\ 
L19-094 &  108 &  34 &  56 &  52 &  300 &  249 &  119 &  115 &  0.78 &  6.7 \\ 
L19-095 &  14 &  $\sim$95 &  $\sim$70 &  $\sim$145 &  300 &  $\sim$406 &  291 &  207 &  1.66 &  7.2 \\ 
L19-096 &  564 &  58 &  -- &  -- &  300 &  181 &  140 &  105 &  0.82 &  6.7 \\ 
L19-097 &  1351 &  -- &  518 &  97 &  300 &  276 &  160 &  153 &  1.04 &  7.8 \\ 
L19-098 &  171 &  53 &  $\sim$120 &  $\sim$15 &  300 &  149 &  116 &  91 &  0.69 &  6.7 \\ 
L19-099 &  61 &  51 &  -- &  -- &  300 &  $\sim$120 &  93 &  52 &  0.48 &  7.6 \\ 
L19-100 &  243 &  58 &  4 &  16 &  300 &  118 &  84 &  57 &  0.47 &  8.8 \\ 
L19-101 &  103 &  -- &  -- &  -- &  300 &  105 &  97 &  69 &  0.55 &  7.1 \\ 
L19-102 &  56 &  $\sim$34 &  86 &  60 &  300 &  298 &  149 &  110 &  0.86 &  6.9 \\ 
L19-103 &  120 &  53 &  134 &  -- &  300 &  96 &  104 &  73 &  0.59 &  8.8 \\ 
L19-105 &  73 &  -- &  129 &  55 &  300 &  217 &  193 &  141 &  1.11 &  6.8 \\ 
L19-107 &  330 &  23 &  -- &  -- &  300 &  127 &  58 &  43 &  0.34 &  6.0 \\ 
L19-108 &  18 &  $\sim$54 &  $\sim$282 &  -- &  300 &  $\sim$246 &  184 &  128 &  1.04 &  8.8 \\ 
L19-110 &  64 &  $\sim$17 &  -- &  50 &  300 &  190 &  120 &  99 &  0.73 &  8.7 \\ 
L19-111 &  258 &  45 &  98 &  35 &  300 &  170 &  116 &  90 &  0.69 &  6.8 \\ 
L19-112 &  31 &  -- &  -- &  $\sim$36 &  300 &  $\sim$158 &  104 &  82 &  0.62 &  6.1 \\ 
L19-113 &  311 &  29 &  9 &  14 &  300 &  150 &  95 &  72 &  0.56 &  6.7 \\ 
L19-117 &  863 &  37 &  -- &  8 &  300 &  116 &  52 &  37 &  0.29 &  7.3 \\ 
L19-120 &  92 &  52 &  $\sim$51 &  $\sim$24 &  300 &  $\sim$105 &  101 &  73 &  0.58 &  9.0 \\ 
L19-122 &  12 &  $\sim$31 &  $\sim$28 &  -- &  300 &  $\sim$239 &  $\sim$186 &  $\sim$133 &  $\sim$1.06 &  6.0 \\ 
L19-123 &  2638 &  34 &  5 &  14 &  300 &  115 &  52 &  44 &  0.32 &  6.2 \\ 
L19-124 &  127 &  $\sim$23 &  24 &  39 &  300 &  135 &  97 &  73 &  0.57 &  6.2 \\ 
L19-125 &  80 &  $\sim$35 &  -- &  -- &  300 &  $\sim$103 &  89 &  60 &  0.50 &  7.2 \\ 
L19-126 &  242 &  44 &  89 &  35 &  300 &  147 &  97 &  80 &  0.59 &  8.0 \\ 
L19-127 &  40 &  -- &  65 &  133 &  300 &  323 &  -- &  -- &  -- &  6.9 \\ 
L19-130 &  60 &  $\sim$34 &  192 &  74 &  300 &  266 &  165 &  145 &  1.03 &  7.9 \\ 
L19-131 &  61 &  $\sim$49 &  -- &  -- &  300 &  155 &  105 &  82 &  0.62 &  7.6 \\ 
L19-132 &  7 &  -- &  $\sim$148 &  $\sim$122 &  300 &  $\sim$171 &  $\sim$215 &  $\sim$138 &  $\sim$1.18 &  5.5 \\ 
L19-134 &  93 &  51 &  46 &  91 &  300 &  139 &  180 &  132 &  1.04 &  8.0 \\ 
L19-135 &  74 &  47 &  -- &  -- &  300 &  129 &  174 &  116 &  0.97 &  7.1 \\ 
L19-137 &  317 &  52 &  55 &  69 &  300 &  168 &  102 &  97 &  0.66 &  6.9 \\ 
L19-139 &  472 &  -- &  -- &  36 &  300 &  102 &  103 &  76 &  0.60 &  7.5 \\ 
L19-140 &  82 &  62 &  115 &  63 &  300 &  173 &  130 &  99 &  0.76 &  7.9 \\ 
L19-141 &  155 &  38 &  $\sim$116 &  $\sim$8 &  300 &  83 &  39 &  29 &  0.23 &  6.7 \\ 
L19-142 &  152 &  60 &  $\sim$23 &  -- &  300 &  89 &  58 &  41 &  0.33 &  8.1 \\ 
L19-146 &  271 &  55 &  11 &  16 &  300 &  112 &  106 &  74 &  0.60 &  8.4 \\ 
\enddata 
\tablenotetext{a}{ H$\alpha$ Flux is in ergs cm$^{-2}$ s$^{-1}$.}
\tablenotetext{b}{ Emission line strengths are listed relative to H$\alpha$ set to 300.}
\tablenotetext{c}{ FWHM is in \AA.}
\label{table_snr_spec}
\end{deluxetable}
%\end{center}


\pagebreak
%% To help institutions obtain information on the effectiveness of their 
%% telescopes the AAS Journals has created a group of keywords for telescope 
%% facilities.
%
%% Following the acknowledgments section, use the following syntax and the
%% \facility{} or \facilities{} macros to list the keywords of facilities used 
%% in the research for the paper.  Each keyword is check against the master 
%% list during copy editing.  Individual instruments can be provided in 
%% parentheses, after the keyword, but they are not verified.

% \begin{figure}
% \plotone{rgb_10arcmin_subtracted_2.png}
%\caption{An emission-line image of \gal, after continuum subtraction, where R = \ha, G = \sii, and B = \oiii.  The field is 10\arcmin\ square and is oriented N up, E left. \label{fig_overview} }
%\end{figure}

\begin{figure}
% \plotone{ngc6946_rgb_coords_1.png}
\plotone{fig01.png}
\caption{An emission-line image of \gal, where R = \ha, G = \sii, and B = \oiii, taken from the 3.8m WIYN telescope at Kitt Peak.  This figure uses the emission line images prior to continuum subtraction, so the underlying galaxy light is also visible. Yellow circles indicate the positions of the ten historical SNe since 1917 (including SN\,2017eaw, which occurred subsequent to our observations). The dashed rectangle indicates the small region shown in detail in Fig.~\ref{fig_diagnostic}.  The field is 10\arcmin\ square. 
\label{fig_rgb_overview}}
\end{figure}

\begin{figure}
% \plotone{ha_sub_index.png}
\plotone{fig02.png}
\caption{Image of \gal\ in \ha, after continuum subtraction to remove most of the stars.  Green circles indicate the positions of SNRs and candidates from MF97; blue circles indicate the positions of our new \sii-selected candidates; yellow circles indicate the positions of historical SNe.  The red squares denote those objects for which we obtained GMOS spectra (Table~4).  The field is identical to that shown in Fig.~\ref{fig_rgb_overview}. \label{fig_index_overview}}
\end{figure}



\begin{figure}
% \plotone{fig_diagnostic_4pan.pdf}
\plotone{fig03.pdf}
% \plotone{f}
\caption{This figure demonstrates the diagnostic process used to find SNR candidates in \gal.  The region shown is 35\arcsec\ in the N-S dimension and is centered $\sim$1.5\arcmin\ south of the nucleus (see Fig. 1).  At upper left is a continuum-subtracted \sii\ image for reference.  At upper right, we show a color image of subtracted emission line images, where red is H$\alpha$, green is \sii, and blue is \oiii.  Bottom left shows the \sii:H$\alpha$ ratio image of the region, where elevated values of the  ratio are white and low values are black.  The lower right image shows the green continuum image, which is useful for identifying the presence of stellar subtraction residuals.  The red circles are 4\arcsec\ in diameter and show two MF objects and four newly identified SNR candidates from our survey (identifications shown in upper left panel).  Note the appearance in the ratio image, where most of the emission nebulae show low ratios, but the objects in the red circles stand out in the ratio.  In the upper right panel, SNR candidates appear as greenish compared to photoionized nebulae, due to relatively stronger \sii\ and/or \oiii\ emission. \label{fig_diagnostic}}
\end{figure}

\begin{figure}
% \plotone{example_spectra_n6946.png}
\plotone{fig04.png}
\caption{Examples of the spectra obtained for three representative SNR candidates and one \hii\ region.  The three SNR spectra have been selected to illustrate the quality of the spectra for a bright, medium bright and fairly faint candidate. The traces have been scaled arbitrarily, and offset for clarity. \label{fig_example_spectra}}
\end{figure}

\begin{figure}

% \plottwo{fig_s2_ha_ratio.pdf}{fig_spec_im_comp.pdf}
\plottwo{fig05a.pdf}{fig05b.pdf}
\caption{Left: The \sii:\ha\ ratio obtained spectroscopically for SNR candidates (red) and \hii\ regions (blue) as a function of \ha flux in the spectrum.  Objects with ratios greater than 0.4 (the dashed line) are spectroscopically confirmed SNRs.  Objects near the dividing line are less certain because observational errors in the ratio and/or \hii\ contamination can impact the derived ratio. Right: The \sii:\ha\ ratio derived from spectra compared to that derived from narrow band imaging.  The tendency for spectroscopic ratios to be somewhat higher is consistent with mild impacts from \nii\ emission getting through the \ha\ imaging filter.  \label{fig_s2_ha}}

\end{figure}

\begin{figure}
% \plotone{fig_s2_ratio.pdf}
\plotone{fig06.pdf}
\caption{The [S II]6717:[S II]6731 line ratio for SNR candidates as a function of galactocentric distance.  This ratio is nominally a density diagnostic, and most candidates are near the low density limit of 1.4, indicating generally low ISM densities. Derived values above 1.4 are non-physical, and are indicative of errors in the derived ratio.  Only a handful of objects appear to show densities significantly above the low-density limit.  There is no obvious trend with galactocentric distance.  \label{fig_s2_ratio}}
\end{figure}

\begin{figure}
% \plotone{fig_reddening_gc.pdf}
\plotone{fig07.pdf}
\caption{Observed \hb:\ha\ line ratios for SNR candidates as a function of galactocentric distance. The dashed lines indicate the expected values of the line ratio for E(B-V) of 0.0, 0.1, 0.3 and 1.0.  Objects near the galactic center tend to be more highly reddened than those far from the center. \label{fig_reddening}}
\end{figure}

\begin{figure}
% \plottwo{fig_n2_ha_gc.pdf}{fig_s2_ha_gc.pdf}
\plottwo{fig08a.pdf}{fig08b.pdf}
\caption{Line ratios of [N II]:\ha\ and [S II]:\ha\ as a function of GCD.  Both SNR candidates and \hii\ regions are shown.     \label{fig_metal_ratios}}

\end{figure}

\begin{figure}
% \plotone{fig_s2_n2.pdf}
\plotone{fig09.pdf}
\caption{A comparison of the line rations of [N II]:\ha\ to that of [S II]:\ha\ for the SNR candidates. \label{fig_s2_n2}}
\end{figure}


\begin{figure}
% \plottwo{fig_model_n2o3.pdf}{fig_model_s2n2.pdf}
\plottwo{fig10a.pdf}{fig10b.pdf}
\caption{Left: Model [O III] 5007:\hb\ ratio as function of the \nii\ 6583:\ha\ line ratio for SNRs and SNR candidates with spectra. As discussed in the text, the black, green and blue meshes correspond to shock models from Allen et al. (2008) with a range of shock velocities and pre-shock magnetic fields, and with metallicities corresponding to the SMC (black), LMC (green), and Milky Way (blue).  Right: Model \nii\ 6583:\ha\ line ratios as a function of the \sii:\ha\ ratios.  \gal\ objects appear consistent with solar abundances and a wide range of other physical properties.  \label{fig_model}}
\end{figure}

\begin{figure}
% \plotone{spectra_historical_sne.png}
\plotone{fig11.png}
\caption{GMOS spectra of the two historical SNe we have recovered in \gal.  Both show the broad emission lines characteristic of young SNRs with rapidly expanding ejecta.  (The spectra have been displaced vertically for clarity.) \label{historical_SNe}}
\end{figure}


%% Appendix material should be preceded with a single \appendix command.
%% There should be a \section command for each appendix. Mark appendix
%% subsections with the same markup you use in the main body of the paper.

%% Each Appendix (indicated with \section) will be lettered A, B, C, etc.
%% The equation counter will reset when it encounters the \appendix
%% command and will number appendix equations (A1), (A2), etc. The
%% Figure and Table counter will not reset.

%\appendix



%% The reference list follows the main body and any appendices.
%% Use LaTeX's thebibliography environment to mark up your reference list.
%% Note \begin{thebibliography} is followed by an empty set of
%% curly braces.  If you forget this, LaTeX will generate the error
%% "Perhaps a missing \item?".
%%
%% thebibliography produces citations in the text using \bibitem-\cite
%% cross-referencing. Each reference is preceded by a
%% \bibitem command that defines in curly braces the KEY that corresponds
%% to the KEY in the \cite commands (see the first section above).
%% Make sure that you provide a unique KEY for every \bibitem or else the
%% paper will not LaTeX. The square brackets should contain
%% the citation text that LaTeX will insert in
%% place of the \cite commands.

%% We have used macros to produce journal name abbreviations.
%% \aastex provides a number of these for the more frequently-cited journals.
%% See the Author Guide for a list of them.

%% Note that the style of the \bibitem labels (in []) is slightly
%% different from previous examples.  The natbib system solves a host
%% of citation expression problems, but it is necessary to clearly
%% delimit the year from the author name used in the citation.
%% See the natbib documentation for more details and options.
% \nocite{}
%\begin{thebibliography}{}

%\bibliography{snr}

%\end{thebibliography}

%% This command is needed to show the entire author+affilation list when
%% the collaboration and author truncation commands are used.  It has to
%% go at the end of the manuscript.
%\allauthors

%% Include this line if you are using the \added, \replaced, \deleted
%% commands to see a summary list of all changes at the end of the article.
%\listofchanges

\end{document}

% End of file `sample61.tex'.
